\documentclass[letterpaper,spanish]{book}

\usepackage[T1]{fontenc}
\usepackage[utf8]{inputenc}
\setcounter{tocdepth}{1}
\usepackage{babel}
\usepackage[
    style=authoryear,
    doi=false,
    isbn=false,
    url=false,
    eprint=false,
    dashed=false,
    uniquename=init
]{biblatex}
\usepackage{csquotes}

\usepackage{float}
\usepackage{booktabs}
\usepackage{enumitem}
\usepackage[algochapter,vlined,linesnumbered,ruled]{algorithm2e}
\usepackage{amsmath}
\usepackage{amsthm}
\usepackage{amssymb}
\usepackage{graphicx}
\usepackage{multicol}
\usepackage[
    unicode=true,
    pdfusetitle,
    bookmarks=true,
    bookmarksnumbered=true,
    bookmarksopen=false,
    breaklinks=false,
    pdfborder={0 0 1},
    backref=false,
    colorlinks=false
]{hyperref}

%% Definición de ambientes de uso matemático
\providecommand{\casename}{Caso}
\providecommand{\corollaryname}{Corolario}
\providecommand{\definitionname}{Definición}
\providecommand{\examplename}{Ejemplo}
\providecommand{\lemmaname}{Lema}
\providecommand{\propositionname}{Proposición}
\providecommand{\remarkname}{Observación}
\providecommand{\theoremname}{Teorema}

\theoremstyle{definition}
    \ifx\thechapter\undefined
      \newtheorem{defn}{\protect\definitionname}
    \else
      \newtheorem{defn}{\protect\definitionname}[chapter]
    \fi
\theoremstyle{plain}
    \ifx\thechapter\undefined
      \newtheorem{prop}{\protect\propositionname}
    \else
      \newtheorem{prop}{\protect\propositionname}[chapter]
    \fi
\theoremstyle{plain}
    \ifx\thechapter\undefined
	    \newtheorem{thm}{\protect\theoremname}
	  \else
      \newtheorem{thm}{\protect\theoremname}[chapter]
    \fi
\theoremstyle{remark}
    \ifx\thechapter\undefined
      \newtheorem{rem}{\protect\remarkname}
    \else
      \newtheorem{rem}{\protect\remarkname}[chapter]
    \fi
\theoremstyle{plain}
    \ifx\thechapter\undefined
  \newtheorem{cor}{\protect\corollaryname}
\else
      \newtheorem{cor}{\protect\corollaryname}[chapter]
    \fi
\theoremstyle{plain}
    \ifx\thechapter\undefined
      \newtheorem{lem}{\protect\lemmaname}
    \else
      \newtheorem{lem}{\protect\lemmaname}[chapter]
    \fi
\theoremstyle{definition}
    \ifx\thechapter\undefined
      \newtheorem{example}{\protect\examplename}
    \else
      \newtheorem{example}{\protect\examplename}[chapter]
    \fi

%% Configuración del enumerado de diferentes ambientes matemáticos
\newlist{casenv}{enumerate}{4}
\setlist[casenv]{leftmargin=*,align=left,widest={iiii}}
\setlist[casenv,1]{label={{\itshape\ \casename} \arabic*.},ref=\arabic*}
\setlist[casenv,2]{label={{\itshape\ \casename} \roman*.},ref=\roman*}
\setlist[casenv,3]{label={{\itshape\ \casename\ \alph*.}},ref=\alph*}
\setlist[casenv,4]{label={{\itshape\ \casename} \arabic*.},ref=\arabic*}

%% Personalización del formato de los títulos
\usepackage{titlesec}
\titleformat{\chapter}[display]{\bfseries\centering\large}{\normalsize{\chaptertitlename\ \thechapter}}{0.5ex}{}[]
\titlespacing*{\chapter}{0pt}{-50pt}{20pt}
\titleformat{\section}[hang]{\bfseries\large}{\thesection\ }{0pt}{}[]
\titleformat{\subsection}[hang]{\bfseries\normalsize}{\thesubsection\ }{0pt}{}[]
\titleformat{\paragraph}[runin]{\itshape\normalsize}{\theparagraph\ }{0pt}{}[]

%% Eliminar el espacio que hay entre cada elemento de una lista
\usepackage{enumitem}
\setlist{nolistsep}

%% Declaración de palabras clave y configuración de estilos del pseudocódigo
\SetKw{Error}{error}
\SetKw{And}{and}
\SetKw{Or}{or}
\SetKw{Not}{not}
\SetKw{True}{true}
\SetKw{False}{false}
\SetKw{Nil}{nil}
\SetKw{Down}{down to}
\SetCommentSty{textit}
\SetKwComment{Comment}{$\triangleright$ }{}
\SetFuncSty{textsc}
\DontPrintSemicolon

%% Símbolo para indicar el final de un ejemplo
\newcommand\xqed[1]{\leavevmode\unskip\penalty9999 \hbox{}\nobreak\hfill\quad\hbox{#1}}
\newcommand\exend{\xqed{$\blacksquare$}}

%% Eliminar los encabezados de las páginas de los capítulos
\renewcommand{\chaptermark}[1]{\markboth{ }{}}
\renewcommand{\sectionmark}[1]{\markright{ }{}}

%% Se eliminan algunos campos innecesarios de las ref. bibliográficas
\AtBeginDocument{
    \AtEveryBibitem{\clearfield{month}}
    \AtEveryBibitem{\clearfield{day}}
    \AtEveryBibitem{\clearfield{note}}
    \AtEveryBibitem{\clearlist{location}}
    \AtEveryBibitem{\clearfield{eventtitle}}
    \DeclareFieldFormat[article,inbook,incollection,inproceedings,patent,thesis,unpublished]{citetitle}{#1}
    \DeclareFieldFormat[article,inbook,incollection,inproceedings,patent,thesis,unpublished]{title}{#1} 
}


\addbibresource{apuntes.bib}

\begin{document}
    \title{ALGORÍTMICA}
    \author{Carlos A. Oliva Moreno}
    \maketitle
    
    \tableofcontents{}
    
    \part{Algoritmos Secuenciales}
        \chapter{Conceptos fundamentales}

Un \emph{problema computacional} es una relación entre dos valores o colecciones de valores: uno de \emph{entrada} (input) y otro de \emph{salida} (output). Por ej., el problema de \textsc{Ordenamiento} se define formalmente de la sig. manera:
\begin{itemize}
  \item \emph{Entrada}: una secuencia de \(n\in\mathbb{N}\) elementos comparables, \(A=\{a_1,a_2,\dots,a_n\}\).
  \item \emph{Salida}: una permutación de la secuencia de entrada, \(A'=\{a'_1,a'_2,\dots,a'_n\}\), tal que \(a'_1\leq a'_2\leq\dots\leq a'_n\).
\end{itemize}

Un \emph{caso específico} (instance) para un problema computacional determinado es cualquier valor o colección de valores que satisfacen la descripción de la entrada del problema. 
Por ej., dos casos específicos para el problema de \textsc{Ordenamiento} son las secuencias \(\{4,7,5,1\}\) y \(\{d,x,j,e\}\).

Informalmente\marginnote{Realmente no existe una definición formal y uniforme sobre qué es un algoritmo.}, un \emph{algoritmo} es un procedimiento inambiguo para transformar la entrada de un problema computacional determinado a la salida correspondiente.
Se dice que un algoritmo es \emph{correcto} cuando, para todo caso específico, termina su ejecución y produce un resultado que cumple la descripción de la salida del problema.
Más formalmente, un algoritmo es \emph{totalmente correcto} si y solo si satisface las sig. condiciones: \marginnote{Casi siempre es trivial demostrar la terminación de un algoritmo, por lo que, en la práctica, esta parte suele omitirse y únicamente se demuestra la corrección parcial del algoritmo.}
\begin{enumerate}
  \item \emph{Terminación}: el algoritmo termina su ejecución para todo caso específico.
  \item \emph{Correctitud parcial}: dado un caso específico determinado, si el algoritmo termina su ejecución, entonces produce un resultado que satisface la descripción de la salida del problema.
\end{enumerate}
Se dice que un algoritmo correcto \emph{resuelve} el problema en cuestión.

Una \emph{estructura de datos} es una colección de reglas y procedimientos para organizar, accesar y manipular una colección de datos.

Por último, \marginnote{Hay muchas otras características que podrían ser de interés, dependiendo de la aplicación del algoritmo que se está analizando. Por ej., una característica frecuentemente estudiada es el orden de crecimiento de la cantidad de memoria ocupada por el algoritmo con respecto al tamaño de la entrada.} el \emph{análisis de algoritmos} es la colección de técnicas y herramientas matemáticas que se usan para caracterizar las propiedades particulares de un algoritmo determinado de forma independiente a su implementación en hardware y/o software. 
Por ahora, el análisis de un algoritmo consistirá de describir únicamente dos características:

\begin{itemize}
  \item La \emph{corrección}; esto es, ¿el algoritmo es correcto?
  \item La \emph{eficiencia}; esto es, ¿cuál es el orden de crecimiento del 
  tiempo de ejecución del algoritmo con respecto al tamaño de la entrada?
\end{itemize}
Se dice que un algoritmo es \emph{eficiente} si el orden de crecimiento
de su tiempo de ejecución es polinomial.


\section*{Literatura consultada}

\begin{itemize}
  \item \textcite{cormen_introduction_2009}, págs. 5-14 y 20-22.
  \item \textcite{skiena_algorithm_2011}, págs. 3-13.
\end{itemize}

        \chapter{Análisis de la Exactitud de un Algoritmo}

El primer paso al analizar un algoritmo es determinar si éste es correcto
o incorrecto. El procedimiento que se sigue para llevar a cabo esta
tarea se denomina \emph{análisis de la exactitud de un algoritmo}.

\section{Contraejemplo}

Para demostrar que un algoritmo es incorrecto, basta con producir
un \emph{contraejemplo}; i.e. un caso específico para el cual el algoritmo
no termina su ejecución o no produce un resultado que cumple las características
de salida del problema. 

Para producir un contraejemplo, se deben explorar todas las clases
de entrada posibles. Se recomienda comenzar por aquellas que representen
las situaciones más complicadas o inconvenientes para el algoritmo. Por ejemplo, si
se trata de un algoritmo voraz, se puede proponer un caso donde todos
los valores están empatados. Para otros tipos de algoritmos, se pueden
proponer casos que contengan mezclados valores de frontera de extremos
opuestos. Por ejemplo, casos con valores muy grandes y muy pequeños,
muy cerca y muy lejos, muchos y muy pocos, etc.

El contraejemplo debe ser lo más simple y pequeño posible; idealmente,
la razón de por qué el algoritmo es incorrecto debe ser inmediatamente
clara. Así, una vez encontrado un contraejemplo, se recomienda simplificarlo
tanto como sea posible. Además, el contraejemplo debe presentarse
junto con el resultado incorrecto que produjo el algoritmo y el resultado
correcto que debió producir.

Es importante destacar que, si no se logra encontrar un contraejemplo
para un algoritmo determinado, esto no implica que dicho algoritmo
es correcto; aún es necesario demostrar su exactitud.

\section{La invariante de lazo}

Una \emph{invariante de lazo} (loop invariant; abreviado como IL) es una proposición
lógica que se cumple inmediatamente antes e inmediatamente después
de cada iteración de un bucle o algoritmo iterativo determinado. La
IL siempre se define en función del número de iteraciones.

Para producir una IL, se recomienda primero hacer
una ``prueba de escritorio'' del bucle; i.e. ejecutar cada paso
del algoritmo en una hoja de papél con uno o varios casos específicos
de prueba. Esto permite identificar patrones en el comportamiento
de dicho bucle que podrían aprovecharse para definir la IL. 

Demostrar que una IL se cumple para cualquier iteración
es un procedimiento análogo a una demostración por inducción y consiste
de los sig. pasos: 

\begin{enumerate}
    \item \emph{Inicialización}: se considera cuál es el estado que se tiene
    antes de ejecutar la primera iteración y se demuestra que la IL se cumple bajo estas condiciones.
    \item \emph{Mantenimiento}: se supone que la IL se cumple
    antes de comenzar alguna iteración genérica $i$ y se determina el
    estado que se obtiene como consecuencia de ello. Después, se ejecuta
    la iteración $i$ y se demuestra que la IL se cumple antes
    de comenzar la iteración $i+1$.
    \item \emph{Finalización}: se determina cuál es el estado al salir del bucle
    y se utiliza la IL para demostrar que el algoritmo
    es correcto o para caracterizar alguna propiedad particular del algoritmo. 
\end{enumerate}

\section{Cómo diseñar algoritmos correctos}

El diseño de un algoritmo es un proceso iterativo; rara vez se puede
diseñar un algoritmo correcto en el primer intento. A continuación
se presentan algunos consejos para diseñar algoritmos correctos. 

\subsection{Definir el problema correctamente}

Si el problema no está bien definido, será muy difícil o incluso imposible
diseñar un algoritmo que lo resuelva. 

Las características de salida
deben ser claras y no deben consistir de objetivos compuestos. Por
ejemplo, decir ``encontrar la mejor ruta entre dos puntos en un mapa'',
es un problema mal definido, pues no queda claro a qué se refiere
exactamente con ``la mejor ruta''. Otro ejemplo, decir ``encontrar
la ruta más corta entre dos puntos en un mapa que además requiera
menos del doble de los giros a la izquierda de los que son mínimamente
necesarios'', es un problema bien definido pero muy difícil de resolver
pues la salida consiste de cumplir varios objetivos intermedios. 

En cuanto a las características de entrada, típicamente entre más
restrictivas sean, menor es la dificultad para resolver el problema.
Por ende, se recomienda comenzar por diseñar un algoritmo correcto
que admita entradas de características muy particulares y después
extenderlo a entradas más generales. Por ejemplo, en lugar de trabajar
con grafos generales, se puede trabajar primero con árboles.

\subsection{Analizar la complejidad computacional del problema}

Se recomienda analizar la complejidad computacional del problema al
mismo tiempo que se diseña un algoritmo para resolverlo. De esta forma,
cualquier descubrimiento o avance que se obtenga por un lado podría
aprovecharse después para avanzar en el otro. Además, al final se
obtendría uno de dos posibles resultados: ya sea se llega a la construcción
de un algoritmo correcto o se caracteriza la complejidad computacional
del problema.

\subsection{Elegir un modelo adecuado}

Las entidades con las que interactúa un problema en la vida real suelen
representarse por medio de alguna de varias estructuras abstractas
ya conocidas (e.g. cadenas, grafos, conjuntos, etc.). Sin embargo,
las especificaciones del problema no siempre se ajustan perfectamente
a las características de la estructura elegida. En estos casos, se
recomienda ignorar temporalmente aquellos detalles que no encajen
y decidir más adelante, después de trabajar un tiempo con esa estructura,
si dichos detalles son realmente esenciales o no para resolver el
problema. 

\section*{Notas bibliográficas}

\begin{itemize}
    \item \textcite{cormen_introduction_2009}, págs. 18-20. 
    \item \textcite{skiena_algorithm_2011}, págs. 11-16.
\end{itemize}

        
\chapter{La Notación Asintótica}

La notación asintótica (también conocida como la \emph{notación de
Landau}) es una notación matemática que se utiliza para caracterizar,
por medio de una cota, la tasa de crecimiento de una función cuando
la variable independiente tiende a infinito.

\section{Definiciones básicas}

\marginpar{%
    \includegraphics[width=\marginparwidth]{figuras/big-theta}
    \captionof{figure}{{\small{}Interpretación gráfica de la notación 
    asintótica. La notación $f=\Theta(g)$
    implica que $g$ es una cota ajustada para $f$.}}
    \includegraphics[width=\marginparwidth]{figuras/big-o}
    \captionof{figure}{{\small{}La notación $f=O(g)$
    implica que $g$ es una cota superior para $f$.}}
    \includegraphics[width=\marginparwidth]{figuras/big-omega}
    \captionof{figure}{{\small{}La notación $f=\Omega(g)$ 
    implica que $g$ es una cota inferior para $f$.}}
}

Sea $n\in\mathbb{N}$ y sean $f:\mathbb{N}\to\mathbb{R}$ y $g:\mathbb{N}\to\mathbb{R}$
dos funciones \emph{asintóticamente positivas}; i.e. $f(n)$ y $g(n)$
siempre son positivos a partir de algún valor de $n$. A continuación
se presentan las definiciones básicas para la notación asintótica.
Obsérvese que, en la notación asintótica, se utiliza el símbolo $=$
en lugar de $\in$ para denotar que una función determinada pertenece
a alguno de los conjuntos introducidos.

\begin{defn}[O grande]
    Denotado como $O(g)$, es el conjunto de todas aquellas funciones
    $f$ para las cuales existen dos constantes $c\in\mathbb{R}$ y $n_{0}\in\mathbb{N}$
    tales que $0\le f(n)\leq c g(n)$, para toda $n\geq n_{0}$. 
\end{defn}

\begin{defn}[$\Omega$ grande]
    Denotado como $\Omega(g)$, es el conjunto de todas aquellas funciones
    $f$ para las cuales existen dos constantes $c$ y $n_{0}$ tales
    que $0\le c g(n)\leq f(n)$, para toda $n\geq n_{0}$.
\end{defn}

\begin{defn}[$\Theta$ grande]
    Denotado como $\Theta(g)$, es el conjunto de todas aquellas funciones
    $f$ para las cuales existen tres constantes $c_{1},c_{2}\in\mathbb{R}$
    y $n_{0}$ tales que $0\le c_{1} g(n)\leq f(n)\leq c_{2} g(n)$,
    para toda $n\geq n_{0}$.
\end{defn}

\begin{prop}
    Se tiene que $f=\Theta(g)$ si y sólo si $f=O(g)$ y $f=\Omega(g)$.
\end{prop}

\begin{defn}[o chica]
    Denotado como $o(g)$, es el conjunto de todas aquellas funciones
    $f$ tales que, para toda constante $c$, existe una constante $n_{0}$
    tal que $0\leq f(n)<c g(n)$ para toda $n\geq n_{0}$.
\end{defn}

\begin{defn}[$\omega$ chica]
    Denotado como $\omega(g)$, es el conjunto de todas aquellas funciones
    $f$ tales que, para toda constante $c$, existe una constante $n_{0}$
    tal que $0\leq c g(n)<f(n)$ para toda $n\geq n_{0}$.
\end{defn}

\begin{prop}
    Si $f=o(g)$, entonces $f=O(g)$.
\end{prop}

\begin{prop}
    Si $f=\omega(g)$, entonces $f=\Omega(g)$.
\end{prop}

\section{Definiciones a partir de límites}

La notación asintótica también se puede definir en términos del límite
de la razón de las funciones involucradas (suponiendo que dicho límite
existe). 

\begin{prop}
    Se tiene que
    
    \[
        \lim_{n\to\infty}\dfrac{f(n)}{g(n)}\in\begin{cases}
            [0,\infty) & \text{si y sólo si }f=O(g)\\
            (0,\infty] & \text{si y sólo si }f=\Omega(g)\\
            (0,\infty) & \text{si y sólo si }f=\Theta(g)
        \end{cases}
    \]
\end{prop}

\begin{prop}
    Se tiene que

    \[
        \text{si }\lim_{n\to\infty}\dfrac{f(n)}{g(n)}=\begin{cases}
        0 & \text{entonces }f=o(g)\\
        \infty & \text{entonces }f=\omega(g)\\
        1 & \text{entonces }f\sim g
        \end{cases}
    \]
\end{prop}

\begin{prop}
    Si $f\sim g$, entonces $f=\Theta(g)$.
\end{prop}

\section{Ordenes de crecimiento}

La notación asintótica permite clasificar las funciones según su tasa
de crecimiento. Tal clasificación se denomina \emph{orden de crecimiento}
(o, simplemente, \emph{orden}). A continuación se presentan los órdenes
de crecimiento más comunes, listados de aquél que crece más lento
al que crece más rápido.

\begin{enumerate}
    \item \emph{Constantes}: $O(1)$
    \item \emph{Logarítmicos}: $O(\log n)$
    \item \emph{Radicales}: $O(\sqrt{n})$
    \item \emph{Lineales}: $O(n)$
    \item \emph{Súper lineales}: $O(n\log n)$
    \item \emph{Cuadráticos}: $O(n^{2})$
    \item \emph{Cúbicos}: $O(n^{3})$
    \item \emph{Exponenciales}: $O(2^{n})$
    \item \emph{Factoriales}: $O(n!)$
\end{enumerate}

Se debe tener cuidado sobre cómo interpretar la lista anterior, ya
que la notación asintótica puede ocultar constantes muy grandes y 
dejar una impresión equivocada de la tasa de crecimiento
de una función comparada con otra. 
Por ejemplo, supóngase que se tiene que $f(n)=10^{100}n=O(n)$
y que $g(n)=10n\cdot\log n=O(n\log n)$. A pesar de que la notación
asintótica indica que $g$ crece más rápido que $f$, en realidad
se trata de lo opuesto, porque la constante $10^{100}$ supera por mucho 
la tasa de crecimiento de $g$.

\begin{thm}
    Sea $k\in\mathbb{N}_{0}$ una constante. Todo polinomio de grado $k$
    pertenece a $\Theta(n^{k})$.
\end{thm}

\begin{proof}
    Sea $f(n)=a_{k}n^{k}+a_{k-1}n^{k-1}+\dots+a_{1}n+a_{0}$ un polinomio
    arbitrario, donde $a_k,\dots,a_{0}\in\mathbb{R}$ son constantes.
    Por la definición de $\Theta$ grande, se requiere demostrar que $f=O(n^{k})$
    y que $f=\Omega(n^{k})$. 
    
    Para demostrar que $f=O(n^{k})$, se tiene que 
    
    \begin{align*}
        a_{k}n^{k}+a_{k-1}n^{k-1}+\dots+a_{1}n+a_{0} &= (a_{k}+a_{k-1}/n+\dots+a_{1}/n^{k-1}+a_{0}/n^{k})n^{k}\\
        &\leq(a_{k}+a_{k-1}+\dots+a_{1}+a_{0})n^{k}
    \end{align*}
    lo que se cumple para toda $n\geq1$. Por lo tanto, $f=O(n^{k})$.
    
    Para demostrar que $f=\Omega(n^{k})$, se tiene que 
    
    \[
        (a_{k}+a_{k-1}/n+\dots+a_{1}/n^{k-1}+a_{0}/n^{k})n^{k}\geq n^{k}
    \]
    lo que se cumple para toda $n\geq1$. Por lo tanto, $f=\Omega(n^{k})$.
\end{proof}

\begin{thm}
    Toda función polinomial crece más rápido que cualquier función logarítmica y
    toda función exponencial crece más rápido que cualquier función polinomial.
    Esto es, sean $k,q\in\mathbb{N}$ constantes, se tiene que $(\log n)^k=o(n^q)$
    y $n^k=o((1+q)^n)$.
\end{thm}

\begin{proof}
    Ambas relaciones se pueden demostrar utilizando límites y la regla
    de L'Hopital. Así, para la primera expresión, se tiene que
    
    \begin{align*}
        \lim_{n\to\infty}\dfrac{(\ln n)^{k}}{n^{q}} &= \left(\lim_{n\to\infty}\dfrac{\ln n}{n^{q/k}}\right)^{k}\\
        &\overset{\text{H}}{=}\left(\lim_{n\to\infty}\dfrac{1/n}{(q/k)n^{q/k-1}}\right)^{k}\\
        &= \left(\lim_{n\to\infty}\dfrac{1}{(q/k)n^{q/k}}\right)^{k}\\
        &= 0
    \end{align*}
    Por lo tanto, $(\log n)^{k}=o(n^{q})$.
    
    Para la segunda expresión, se tiene que 
    
    \begin{align*}
    \lim_{n\to\infty}\dfrac{n^{k}}{(1+q)^{n}} &= \left(\lim_{n\to\infty}\dfrac{n}{(1+q)^{n/k}}\right)^{k}\\
    &\overset{\text{H}}{=}\left(\lim_{n\to\infty}\dfrac{1}{(n/k)(1+q)^{n/k-1}}\right)^{k}\\
    &= 0
    \end{align*}
    Por lo tanto, $n^{k}=o((1+q)^{n})$.
\end{proof}

\section{Propiedades aritméticas}

A continuación se describe cómo hacer aritmética con la notación asintótica.

\subsection{Adición}

La suma de las cotas de dos funciones está gobernada por la función
dominante.

\[
\begin{aligned}
    O(f)+O(g) &= O(\max\{f,g\})\\
    \Omega(f)+\Omega(g) &= \Omega(\max\{f,g\})\\
    \Theta(f)+\Theta(g) &= \Theta(\max\{f,g\})\\
    o(f)+o(g) &= o(\max\{f,g\})\\
    \omega(f)+\omega(g) &= \omega(\max\{f,g\})
\end{aligned}
\]

\subsection{Multiplicación}

La multiplicación de una función con una constante no afecta el comportamiento
asintótico de dicha función.

\[
\begin{aligned}
    O(c f) &= O(f)\\
    \Omega(c f) &= \Omega(f)\\
    \Theta(c f) &= \Theta(f)\\
    o(c f) &= o(f)\\
    \omega(c f) &= \omega(f)
\end{aligned}
\]

Cuando dos funciones se multiplican, ambas contribuyen por igual al
comportamiento asintótico de la función resultante.

\[
\begin{aligned}
    O(f)\cdot O(g) &= O(f\cdot g)\\
    \Omega(f)\cdot\Omega(g) &= \Omega(f\cdot g)\\
    \Theta(f)\cdot\Theta(g) &= \Theta(f\cdot g)\\
    o(f)\cdot o(g) &= o(f\cdot g)\\
    \omega(f)\cdot\omega(g) &= \omega(f\cdot g)
\end{aligned}
\]


\section{Comparación de funciones}

La notación asintótica se puede utilizar como un esquema para comparar
dos funciones, similar a como se puede hacer con números reales. Una
forma intuitiva de verlo es trazando las sig. analogías:

\begin{center}
\begin{tabular}{cc}
    \toprule 
        Números reales & Notación asintótica\tabularnewline
    \midrule
        $a\leq b$ & $f=O(g)$\tabularnewline
        $a\ge b$ & $f=\Omega(g)$\tabularnewline
        $a=b$ & $f=\Theta(g)$\tabularnewline
        $a<b$ & $f=o(g)$\tabularnewline
        $a>b$ & $f=\omega(g)$\tabularnewline
    \bottomrule
\end{tabular}
\par\end{center}

\subsection{Transitividad}

Sea $h:\mathbb{N}\to\mathbb{R}$ una función asintóticamente positiva. 

\begin{itemize}
    \item Si $f=\Theta(g)$ y $g=\Theta(h)$, entonces $f=\Theta(h)$.
    \item Si $f=O(g)$ y $g=O(h)$, entonces $f=O(h)$. 
    \item Si $f=\Omega(g)$ y $g=\Omega(h)$, entonces $f=\Omega(h)$.
    \item Si $f=o(g)$ y $g=o(h)$, entonces $f=o(h)$.
    \item Si $f=\omega(g)$ y $g=\omega(h)$, entonces $f=\omega(h)$.
\end{itemize}

\subsection{Reflexividad}

Las sig. igualdades se satisfacen porque $f(n)=f(n)$ para toda $n$:

\[
    \begin{aligned}
        f &= \Theta(f)\\
        f &= O(f)\\
        f &= \Omega(f)
    \end{aligned}
\]


\subsection{Simetría y simetría traspuesta}

\begin{itemize}
    \item $f=\Theta(g)$ si y sólo si $g=\Theta(f)$.
    \item $f=O(g)$ si y sólo si $g=\Omega(f)$.
    \item $f=o(g)$ si y sólo si $g=\omega(f)$.
\end{itemize}

\subsection{Carencia de tricotomía}

La tricotomía es la propiedad de los números reales donde, dados dos
números $a$ y $b$, siempre se cumple exactamente una de las sig.
condiciones: $a<b$, $a=b$ o $a>b$. La notación asintótica carece
de esta propiedad, pues se puede dar el caso donde no se cumple
que $f=O(g)$ ni que $f=\Omega(g)$. Por ejemplo, si $f(n)$ oscila
periódicamente en lugar de siempre mantener una tendencia a la alza
o a la baja.

\section*{Notas bibliográficas}

Material consultado:
\begin{itemize}
    \item \textcite{cormen_introduction_2009}, págs. 43-52.
    \item \textcite{skiena_algorithm_2011}, págs. 34-41.
    \item \textcite{goodrich_algorithm_2001}, págs. 13-20.
    \item \textcite{baker_2013}
    \item \textcite{leighton_and_rubinfeld_2004}
    \item \textcite{tomescu_2014}
\end{itemize}

        \chapter{Análisis de la eficiencia de un algoritmo}

Después de demostrar que un algoritmo es correcto, el sig. paso es caracterizar su tiempo de ejecución. 
El procedimiento que se sigue para realizar esta tarea se denomina \textbf{análisis de la eficiencia de un algoritmo}. 
En este contexto, el \emph{tiempo de ejecución} de un algoritmo se define como la cantidad total de instrucciones que se ejecutan en función del \emph{tamaño de la entrada}. 
El significado de ``tamaño de la entrada'' depende del contexto del problema que se está tratando.
P. ej., si la entrada es un arreglo, el tamaño es el número de elementos en dicho arreglo, pero si la entrada es un grafo, entonces el tamaño es el número de vértices y el número de aristas. 

\section{El modelo de cómputo RAM}

Un \textbf{modelo de cómputo} es una representación simplificada de alguna tecnología de cómputo particular y funge como una ``máquina abstracta'' donde se puede simular mentalmente la ejecución de un algoritmo para analizar su eficiencia. 
El modelo debe ser lo suficientemente simple para facilitar el análisis y lo suficientemente cercano a la tecnología que representa para que refleje lo más fielmente posible el comportamiento que tendría el algoritmo de ser implementado en dicha tecnología. Existen varios modelos de cómputo, pero el más utilizado para analizar algoritmos secuenciales es el \textbf{modelo RAM} (Random Access Machine). 

El modelo RAM consta de una unidad de procesamiento central y de una unidad de memoria de acceso aleatorio, conectados por algún medio.
La memoria es conceptualmente idéntica a un arreglo; cada casilla puede referenciarse por medio de una \emph{dirección} única y puede almacenar exactamente una \emph{palabra} binaria de tamaño fijo.
Se supone que la palabra es suficientemente larga para representar cualquier valor primitivo, pero también se supone que su longitud está acotada por una constante (de lo contrario, se podría procesar una cantidad irrealísticamente grande de datos en tiempo constante).
También se cuenta con una cantidad constante de \emph{registros}, que son casillas de memoria reservadas que se utilizan para controlar el flujo del programa y almacenar resultados intermedios.
Finalmente, se supone que la entrada del programa se proporciona ya almacenada en la memoria y que la salida debe escribirse también en la memoria.
\newpage

El \textbf{ciclo de máquina} del modelo RAM consta de tres pasos: 
\begin{enumerate}
  \item Se lee algún dato de la memoria. 
  \item Se ejecuta alguna instrucción sobre ese dato.
  \item Se escribe el resultado en la memoria. 
\end{enumerate}
Cada ciclo de máquina se ejecuta en una cantidad constante de tiempo.

En términos prácticos, al simular la ejecución de un algoritmo en
el modelo RAM, se deben seguir las sig. suposiciones:
\begin{itemize}
  \item Las instrucciones del algoritmo pueden ejecutarse únicamente de forma secuencial (ya que solo se cuenta con un procesador). 
  \item Todas las operaciones lógicas, aritméticas y de comparación se ejecutan en tiempo constante, con la excepción del exponente, el factorial, la raíz y el logaritmo.
  \item Se cuenta con una cantidad infinita de memoria. 
  \item Todos los accesos a la memoria (ya sea por medio de una variable, un índice o un puntero) se ejecutan en tiempo constante. 
  \item Invocar una sub-rutina requiere tiempo constante, pero el tiempo requerido para ejecutarla depende del tamaño de su entrada.
\end{itemize}

\section{Los casos de entrada}

Los casos específicos de un algoritmo se categorizan en tres grupos diferentes, dependiendo de cómo influyen en el tiempo de ejecución de dicho algoritmo:
\begin{itemize}
  \item \textbf{Mejor caso}: es el conjunto de todos los casos específicos que provocan que el algoritmo ejecute la \emph{menor} cantidad posible de instrucciones, en función del tamaño de la entrada.
  \item \textbf{Peor caso}: es el conjunto de todos los casos específicos que provocan que el algoritmo ejecute la \emph{mayor} cantidad posible de instrucciones, en función del tamaño de la entrada. 
  \item \textbf{Caso promedio}: representa la cantidad promedio de instrucciones
  que el algoritmo ejecuta (en función del tamaño de la entrada) para todos los casos específicos posibles. 
\end{itemize}
En la práctica, se suele estudiar únicamente el peor caso. 
En ocasiones, también se estudia el caso promedio; p. ej., cuando dicho caso ocurre con mayor frecuencia que los demás o cuando se está analizando un algoritmo aleatorio.

\section{Procedimiento general del análisis}

El análisis de la eficiencia de un algoritmo consiste en multiplicar
el tiempo de ejecución de cada instrucción (suponiendo que se ejecuta
en el modelo RAM) por el número de veces que se ejecuta (dada una
entrada genérica de tamaño arbitrariamente grande y perteneciente a uno de los casos descritos anteriormente). 
Al final, se suman estos productos y se obtiene como resultado una función \(T:\mathbb{N}\to\mathbb{N}\) que caracteriza el tiempo de ejecución el algoritmo en proporción al tamaño de la entrada. 
Esta función se suele expresar por medio de su \emph{orden de crecimiento asintótico}.

\marginnote[-1\baselineskip]{%
  \textbf{Literatura consultada}: \textcite{cormen_2009}, pp. 23-29; \textcite{skiena_2012}, pp. 31-34.
}

        \chapter{Relaciones de Recurrencia}

Una \emph{relación de recurrencia} (o, simplemente, \emph{recurrencia})
es una función expresada en términos de sí misma, pero con una entrada
más pequeña. Las recurrencias se presentan al analizar la
eficiencia de algoritmos de tipo \emph{divide y vencerás}. La \emph{forma
cerrada} de una recurrencia es una función equivalente, pero expresada
sin recursividad. \emph{Resolver} una recurrencia implica encontrar
su forma cerrada.

En el contexto de los algoritmos divide y vencerás, las recurrencias
suelen tener la sig. forma general: 

\[
    T(n)=\begin{cases}
        \sum_{i=1}^{p}T(n_{i})+f(n) & \text{para }n>b\\
        g(n) & \text{en caso contrario}
    \end{cases}
\]
donde

\begin{itemize}
    \item $b\in\mathbb{N}$ es la condición de paro (boundary condition); esto
    es, el valor que debe tener $n$ para entrar al caso base
    \item $g:\mathbb{N}\to\mathbb{N}$ es el tiempo de ejecución del caso base
    \item $p\in\mathbb{N}$ es la cantidad de subproblemas en las que se dividió
    la entrada
    \item $n_{i}\in\mathbb{N}$, tal que $n_{i}<n$, es el tamaño de la entrada
    para el subproblema $i$ 
    \item $f:\mathbb{N}\to\mathbb{N}$ es el tiempo de ejecución para dividir
    la entrada en, y/o combinar las soluciones de los, $p$ subproblemas
\end{itemize}

En general, la función $g$ suele ser una función constante, por lo que suele omitirse. 
Esto es porque un algoritmo comúnmente alcanza su condición de paro cuando $n$ es 
es de un tamaño determinado. Por otro lado, los pisos y los techos también suelen 
omitirse. Esto es porque casi siempre se puede suponer que $n$ se puede dividir en 
en partes enteras, sin que ello afecte la cota del tiempo de ejecución.

A continuación se presentan diferentes métodos para resolver recurrencias.

\section{El método del árbol recursivo}

El método del árbol recursivo es un método gráfico e informal que
consiste de dibujar un árbol donde cada sub-árbol representa el
tiempo total requerido para resolver un subproblema en la recurrencia. Al
sumar el tiempo de todos los nodos del árbol, se obtiene la forma
cerrada de la recurrencia. Antes de estudiar el método en sí, es 
importante recordar algunos conceptos sobre árboles. 

Sea $k\in\mathbb{N}_0$, un árbol \textit{k-ario} es un árbol 
enraizado donde cada nodo tiene a lo más $k$ hijos. Se dice que un
árbol $k$-ario es \textit{lleno} si cada nodo
tiene exactamente 0 o $k$ hijos. Un árbol $k$-ario 
\textit{perfecto} es un árbol $k$-ario lleno donde todas las hojas 
están en el mismo nivel. 

La \textit{profundidad} de un nodo es la distancia entre dicho nodo 
y la raíz. Por definición, el nodo raíz tiene una profundidad de 0. 
En un árbol $k$-ario perfecto, el número de vértices a 
profundidad $d$ está dado por $k^d$. Para terminar, la \textit{altura} 
de un árbol es la profundidad más grande de todos los nodos del árbol. 
Así, la altura de un árbol $k$-ario perfecto de $n\in\mathbb{N}$ 
nodos está dado por $\log_k{n}$.

El método del árbol recursivo consiste de los sig. pasos:

\begin{enumerate}
    \item Se desglosa la recurrencia en un árbol de tal forma que la raíz representa
    el tiempo requerido para combinar y/o dividir el problema original,
    cada nodo interno representa el tiempo requerido para combinar y/o
    dividir un subproblema y cada hoja representa el tiempo requerido
    por el caso base. 
    \item Se calcula la cantidad de niveles en el árbol.
    \item Se calcula la cantidad de hojas.
    \item Para cada nivel, se suman los valores de los nodos que contiene.
    \item Se suma el tiempo total de todos los niveles para obtener la
    forma cerrada.
\end{enumerate}
Las expresiones obtenidas en los pasos 2, 3 y 4 
se definen en función de $n$ y la expresión obtenida en el paso 5 normalmente consiste de 
una sumatoria que debe resolverse para obtener la forma cerrada.

Si se tiene una recurrencia la forma $T(n)=aT(n/b)+f(n)$, donde $a,b\in\mathbb{N}$ y 
$f:\mathbb{N}\to\mathbb{N}$ es una función asintóticamente 
positiva, intuitivamente se puede ver que $a$ indica 
la cantidad de hijos que tendrá cada nodo en árbol recursivo, $b$ indica qué tanto se reduce 
el tamaño del problema en cada nivel y $f(n)$ es el valor que tendrá cada nodo en el árbol.

\begin{expl}
    Considérese la recurrencia $T(n)=3T(\lfloor n/4 \rfloor)+\Theta(n^2)$. 
    Antes de comenzar, vale la pena simplificar $T(n)$. Se
    puede suponer que $n$ es un múltiplo de 4, para así ignorar el piso, y se puede 
    sustituir $\Theta(n^2)$ por $cn^2$, donde $c\in\mathbb{R}$ es 
    una constante. Así, se obtiene la recurrencia $T(n)=3T(n/4)+cn^2$. 
    El sig. paso es dibujar el árbol recursivo de esta recurrencia. La figura 
    \ref{recursion-tree} muestra cómo dibujar dicho árbol y cómo obtener la forma cerrada a 
    partir de él.
    
    Una vez dibujado el árbol, se debe calcular la altura y el número de hojas. 
    Dado que el valor de $n$ se reduce en $1/4$ en cada nivel, se requiere iterar 
    $\log_4{n}$ veces para llegar a $n=1$, que constituye la condición de paro de la 
    recurrencia. Así, se tiene que la altura del árbol es $\log_4{n}$ y el número de 
    hojas está dado por $3^{\log_{4}{n}}=n^{\log_{4}{3}}$.
    
    Para calcular el tiempo total de cada nivel, hay que recordar que cada nodo tiene
    tres hijos y que su valor es $cn^2$, donde $n$ se va reduciendo en un factor fijo. 
    Así, se puede deducir que el tiempo total 
    de cada nivel está dado por $(3/16)^i\cdot cn^2$, donde $i\in\mathbb{N}_0$
    representa la profundidad de cada nivel. Obsérvese que esta expresión aplica para todos 
    los niveles excepto el último, pues el tiempo del último nivel depende del caso base. 
    En este ejemplo, el tiempo del último nivel está dado por
    $n^{\log_{4}{3}}T(1)=\Theta(n^{\log_{4}3})$ (suponiendo que $T(1)$ es una constante).
    
    Por último, se suman los tiempos de todos los niveles, cosa que para este ejemplo 
    resulta en la sig. expresión:
    
    \begin{align*}
        T(n)&=\sum_{i=0}^{\log_{4}n-1}\left(\dfrac{3}{16}\right)^{i}\cdot cn^{2}+\Theta(n^{\log_{4}3}) \\
    	&<\sum_{i=0}^{\infty}\left(\dfrac{3}{16}\right)^{i}\cdot cn^{2}+\Theta(n^{\log_{4}3}) \\
    	&=\dfrac{1}{1-3/16}\cdot cn^{2}+\Theta(n^{\log_{4}3}) \\
    	&=\dfrac{16}{13}\cdot cn^{2}+\Theta(n^{\log_{4}3}) \\
    	&=O(n^{2})+O(n^{\log_{4}3}) \\
    	&=O(n^{2})
	\end{align*}
	\exend
\end{expl}

\begin{figure}[H]
\begin{centering}
    \includegraphics[width=0.75\textwidth]{figuras/recursion-tree1}
    \par\end{centering}
    \begin{centering}
    \includegraphics[width=1\textwidth]{figuras/recursion-tree2}
    \par\end{centering}
    \caption{{\small{}\label{recursion-tree}Usando el método del árbol recursivo para
    resolver la recurrencia $T(n)=3T(n/4)+cn^{2}$. (a)-(c) Desgloce
    del árbol, comenzando con la raíz y continuando con los hijos en cada nivel. 
    (d) Calculando el número de nodos en cada nivel y sumándolos para obtener la
    forma cerrada de la recurrencia.}}
\end{figure}

\section{El método maestro}

El método maestro consiste simplemente en aplicar el teorema que se presenta a
continuación.

\begin{thm}[teorema maestro]
    Sean $a,n\in\mathbb{N}$ y $b\in\mathbb{R}$, donde $a$ y $b$ son
    constantes y $b>1$. Sea $f:\mathbb{N}\to\mathbb{N}$ una función
    asintóticamente positiva y sea $c=\log_{b}a$. Si $T:\mathbb{N}\to\mathbb{N}$
    es una recurrencia de la forma $T(n)=aT(n/b)+f(n)$, entonces se puede
    resolver aplicando uno de los sig. casos:
    \begin{enumerate}
        \item Si $f=O(n^{c-\varepsilon})$ para alguna constante real $\varepsilon>0$,
        entonces $T=\Theta(n^{c})$.
        \item Si $f=\Theta(n^{c})$, entonces $T=\Theta(n^{c}\log n)$.
        \item Si $f=\Omega(n^{c+\varepsilon})$ y si, además, $af(n/b)\leq kf(n)$
        para alguna constante real $k<1$ y para todo valor de $n$ pasado
        algún umbral, entonces $T=\Theta(f)$.
    \end{enumerate}
\end{thm}

\begin{rem}
    Para el teorema maestro, el término $n/b$ también se puede interpretar
    como $\lceil n/b\rceil$ o $\lfloor n/b\rfloor$.
\end{rem}

Una forma intuitiva de interpretar el teorema maestro es la siguiente.
La función $f$ representa el costo de combinación y/o división de los subproblemas
en cada nivel del árbol recursivo, mientras que la función $n^c$ representa
el número de hojas en dicho árbol. De estas dos funciones, aquella que creces
más rápido es la solución de la recurrencia. 

En el caso 1, se tiene que $n^c$
es más grande; i.e. el costo de resolver cada subproblema eventualmente
supera el costo de combinación y/o división. Por ende, $n^c$ es la solución de la 
recurrencia. En el caso 3, se tiene lo opuesto; $f$ crece más rápido que $n^c$.
Para aplicar este caso, $f$ debe además satisfacer lo que se conoce como
la ``condición de regularidad''.
Por último, en el caso 3 se tiene que ambas funciones, $f$ y $n^c$, tienen
la misma tasa de crecimiento y la solución es entonces el orden de crecimiento
de estas funciones multiplicada por un factor logarítmico que representa
la altura del árbol recursivo.

La desventaja del teorema maestro es que no todas las recurrencias se pueden
resolver aplicando este teorema, aunque sí se puede aplicar a la mayoría de 
las recurrencias que ocurren en la práctica.

\begin{expl}
    Sea $T(n)=9T(n/3)+n$. Aplicando el teorema maestro, se tiene que $a=9$, 
    $b=3$ y $f(n)=n$. Así, se obtiene $n^c=n^{\log_{3}{9}}=n^2$. Dado que 
    $f=O(n^{2-\varepsilon})$, donde $\varepsilon=1$, se aplica 
    el caso 1, lo que resulta en $T=\Theta(n^2)$.
    \exend
\end{expl}

\begin{expl}
    Sea $T(n)=T(2n/3)+1$. Aplicando el teorema maestro, se tiene que $a=1$,
    $b=3/2$ y $f(n)=1$. Así, se obtiene $n^c=n^{\log_{3/2}{1}}=n^0=1$.
    Dado que $f=n^c=1$, se tiene que $f=\Theta(n^c)$ y se aplica el caso 2, 
    lo que resulta en $T=\Theta(\log{n})$.
    \exend
\end{expl}

\begin{expl}
    Sea $T(n)=3T(n/4)+n\lg{n}$. Aplicando el teorema maestro, se tiene que 
    $a=3$, $b=4$ y $f(n)=n\lg{n}$. Así, se obtiene 
    $n^c=n^{\log_{4}{3}}=O(n^{0.793})$. Dado que 
    $f=\Omega(n^{\log_{4}{3}+\varepsilon})$,
    donde $\varepsilon\approx 0.2$, se aplica el caso 3 si se demuestra 
    que $f$ cumple la condición de regularidad; i.e. que $af(n/b)\leq kf(n)$,
    donde $k<1$. Para ello, se tiene que
    $3(n/4)\lg(n/4)\leq (3/4)n\lg{n}$ para valores 
    suficientemente grandes de $n$. Como consecuencia,
    se aplica el caso 3, lo que resulta en $T=\Theta(n\log n)$.
    \exend
\end{expl}

\begin{expl}
    Sea $T(n)=2T(n/2)+n\lg{n}$. El teorema maestro no se puede aplicar
    a esta recurrencia a pesar de tener la forma apropiada. Si se 
    maneja que $a=2$, $b=2$ y $f(n)=n\lg{n}$, se obtendría $n^c=n^{\log_{2}{2}}=n$.
    Dado que $f$ es asintóticamente más grande que $n^c$, se podría pensar que se puede
    aplicar el caso 3, pero no es así debido a que $f$ no es
    polinomialmente más grande que $n^c$; i.e. la razón entre $f$ y $n^c$ 
    no resulta en un polinomio.
    \exend
\end{expl}

\section{El método de Akra-Bazzi}

El método de Akra-Bazzi es una generalización del teorema maestro que admite
variables continuas y que permite resolver recurrencias donde los subproblemas 
en un mismo nivel son de diferente tamaño. El método de Akra-Bazzi trabaja 
sobre recurrencias de la sig. forma:

$$
T(x)=
\begin{cases}
    \sum_{i=1}^{k}a_iT(b_ix)+f(x) &\quad \text{si }x>x_0 \\
    \Theta(1) &\quad \text{si }1\leq x \leq x_0
\end{cases}
$$
donde
\begin{itemize}
    \item $x\geq 1$ es un número real
    \item $x_0\in\mathbb{R}$ es una constante tal que $x_0\geq 1/b_i$ y $x_0\geq 1/(1-b_i)$
    para toda $1\leq i\leq k$
    \item $a_i\in\mathbb{R^+}$ es una constante para toda $1\leq i\leq k$
    \item $b_i\in\mathbb{R^+}$ es una constante en el rango $0<b_i<1$ para toda $1\leq i\leq k$.
    \item $k\geq 1$ es una constante entera
    \item $f(x)$ es una función no negativa $f:\mathbb{R}\to\mathbb{R}_0$ que satisface
    la ``condición de crecimiento polinomial''; i.e. existen dos constantes $c_1,c_2\in\mathbb{R^+}$
    tales que, para toda $x\geq 1$, para toda $1\leq i\leq k$ y para toda $u\in\mathbb{R}$
    tal que $b_ix\leq u \leq x$, se tiene que $c_1f(x)\leq f(u)\leq c_2f(x)$
\end{itemize}
La condición de crecimiento polinomial se cumple para $f(x)$ si $|f'(x)|$ está acotada por
arriba por algún polinomio en $x$. Por ejemplo, $f(x)=x^\alpha\lg^\beta{x}$ satisface esta
condición para cualquier par de constantes reales $\alpha$ y $\beta$.

Para resolver la recurrencia $T(x)$, primero se necesita encontrar un número $p\in\mathbb{R}$
tal que $\sum_{i=1}^{k}a_ib_i^p=1$. Este número siempre existe y es único. Así, la solución
de la recurrencia está dada por

\begin{align*}
    T(n)=\Theta\left( x^p \left( 1 + \int_1^x \dfrac{f(u)}{u^{p+1}}du \right) \right)
\end{align*}

En comparación con el teorema maestro, el método de Akra-Bazzi es más difícil de 
aplicar, pero mucho más versátil pues puede aplicarse a una mayor variedad de
recurrencias.

\begin{expl}
    Considérese la recurrencia $T(n)=7/4\cdot T(\lfloor n/2 \rfloor)+T(\lceil 3n/4 \lceil)+n^2$.
    Aplicando el método de Akra-Bazzi, primero se necesita encontrar la constante $p$ tal que $(7/4)(1/2)^p+(3/4)^p=1$. En este caso, se tiene que $p=2$. Así, la solución a la 
    recurrencia está dada por
    \begin{align*}
        T(n)&=\Theta\left( n^2 \left( 1 + \int_1^n \dfrac{u^2}{u^3}du \right) \right) \\
        &= \Theta\left( n^2 \left( 1 + \int_1^n \dfrac{1}{u}du \right) \right) \\
        &= \Theta\left( n^2 \left( 1 + (\ln{n} - \ln{1}) \right) \right) \\
        &= \Theta\left( n^2 \left( 1 + \ln{n} \right) \right) \\
        &= \Theta(n^2 + n^2\ln{n}) \\
        &= \Theta(n^2\log{n})
    \end{align*}
    \exend
\end{expl}

\section{El método de sustitución}

El método de sustitución es un método formal que consiste en resolver
una recurrencia utilizando inducción matemática. Específicamente, este método
consiste de los sig. pasos:
\begin{enumerate}
    \item Proponer una cota asintótica como la solución tentativa
    de la recurrencia.
    \item Utilizar inducción matemática para demostrar que la cota propuesta
    es correcta y para encontrar las constantes de dicha cota.
\end{enumerate}

Este método se puede utilizar para calcular tanto una cota superior
como una inferior. A continuación se presentan algunas consideraciones
y consejos que se deben tener en cuenta al trabajar con el método de sustitución.

\paragraph{Para proponer una buena solución tentativa}
    Se puede utilizar el método del árbol recursivo para obtener una solución
    tentativa y después utilizar el método de sustitución para demostrar
    que dicha solución es correcta o para ajustar la cota.
    Otra alternativa es que, si la recurrencia tiene una forma similar a alguna otra 
    cuya solución ya se conoce, se puede proponer esa solución como tentativa.
    Como último recurso, se pueden proponer dos cotas holgadas, una inferior y 
    una superior, y ajustarlas gradualmente hasta que converjan en la solución correcta. 
\paragraph{Diferencias con la inducción matemática}
    El caso base se realiza hasta el último. La hipótesis inductiva consiste en suponer 
    que la cota se cumple para un subproblema, $T(n_{i})$, mientras que el paso inductivo 
    consiste en sustituir $T(n_{i})$, en la recurrencia original, por la forma exacta 
    de la cota propuesta en la hipótesis inductiva. Este paso da origen al nombre del método.
\paragraph{Cuidado con la notación asintótica} 
    El objetivo del método de sustitución es demostrar algebraicamente que 
    la cota propuesta se cumple de forma exacta. Es incorrecto aplicar la notación 
    asintótica en el paso inductivo para deshacerse de constantes o términos problemáticos.
\paragraph{Afinando la solución} 
    Cuando la cota propuesta es lo más ajustada posible, 
    pero aún así la inducción no converge, en lugar de proponer una cota más holgada
    se puede proponer una nueva hipótesis inductiva cuya única diferencia
    con la anterior es que se le restan los términos de menor grado.

\begin{expl}
    \label{ex1}
    Considérese la recurrencia $T(n)=2T(\lfloor n/2 \rfloor)+n$ y 
    supóngase que se propone $O(n\lg{n})$ como la solución tentativa. Entonces,
    se busca demostrar que $T(n)\leq cn\lg{n}$ para alguna constante $c>0$.
    
    \paragraph{Hipótesis inductiva}
    Supóngase que $T(\lfloor n/2\rfloor)\leq c\lfloor n/2\rfloor\lg\lfloor n/2\rfloor$.
    
    \paragraph{Paso inductivo}
    Sustituyendo la hipótesis inductiva en la recurrencia original, se tiene lo sig.:
    \begin{align*}
        T(n) &\leq2c\lfloor n/2\rfloor\lg\lfloor n/2\rfloor+n \\
    	&\leq cn\lg(n/2)+n \\
    	&=cn\lg n-cn\lg2+n \\
    	&=cn\lg{n}-cn+n \\
    	&\leq cn\lg n
    \end{align*}
    
    Aparentemente, ve que la solución propuesta es la correcta 
    (para toda $c\geq 1$). Sin embargo, hace falta demostrar que esta solución también se 
    cumple para la condición de paro de la recurrencia. Dicha 
    demostración es por construcción; i.e. se debe encontrar un valor para $c$ que 
    satisfaga la cota en la condición de paro. Esto puede llevar a problemas, como se 
    observa a continuación. 
    
    \paragraph{Caso base}
    Supóngase que $T(1)=1$. Aplicando la cota propuesta a la condición de paro, se tiene que 
    $T(1)\leq cn\lg n=c\lg{1}=0$, lo que contradice que $T(1)=1$. Por lo tanto, la solución 
    propuesta falla en el caso base. Sin embargo, hay que recordar que la definición de la 
    notación asintótica requiere que la cota se cumpla únicamente para un valor de $n$ pasado 
    algún umbral $n_{0}$ que uno puede elegir a voluntad. Esto quiere decir que no es necesario 
    utilizar la condición de paro de la recurrencia como el caso base de la inducción. Para este 
    ejemplo, obsérvese que $T(\lfloor2/2\rfloor)=T(\lfloor3/2\rfloor)=T(1)$, por lo que se puede utilizar 
    $T(2)$ y $T(3)$ como el caso base. Esto es equivalente a elegir $n_0=2$, ya que se tiene
    que para toda $n>3$ la recurrencia ya no depende de $T(1)$.
    Aplicando la hipótesis inductiva al nuevo caso base, se obtiene lo sig.:
    
    \begin{align*}
        T(2) & \leq 2c\lg2 & T(3) & \leq 3c\lg3 \\
        2T(\lfloor 2/2 \rfloor) + 2 & \leq 2c & 2T(\lfloor 3/2 \rfloor) + 3 & \leq 3c(1.585) \\
        2T(1) + 2 & \leq2c & 2T(1) + 3 & \leq 3c(1.585) \\
        4 & \leq 2c & 5 & \leq 3c(1.585) \\
        2 & \leq c & \dfrac{5}{4.755} & \leq c
    \end{align*}
    
    Por lo tanto, queda demostrado que $T(n)\leq cn\lg{n}$ para toda $c\geq 2$.
    \exend
\end{expl}

\begin{expl}
    Considérese ahora la recurrencia $T(n)=T(\lfloor n/2\rfloor)+T(\lceil n/2\rceil)+1$ y 
    supóngase que se propone $O(n)$ como la solución tentativa. Se busca demostrar que
    $T(n)\leq cn$ para alguna constante $c>0$.
    
    \paragraph{Hipótesis inductiva}
    Supóngase que $T(\lfloor n/2\rfloor)\leq c\lfloor n/2\rfloor$ y que 
    $T(\lceil n/2\rceil)\leq c\lceil n/2\rceil$.
    
    \paragraph{Paso inductivo}
    Sustituyendo la hipótesis inductiva en la recurrencia original, se tiene lo sig.:
    \begin{align*}
        T(n) &\leq c\lfloor n/2 \rfloor + 1 = cn + 1
    \end{align*}
    Esto sugiere que la solución propuesta no es correcta. 
    
    A veces puede ocurrir que 
    la solución propuesta sí es correcta pero la inducción no converge en ella.
    Intuitivamente, se puede ver que la cota propuesta en este ejemplo sí es correcta,
    pero el problema es que el término $+1$ impide que se demuestre que la desigualdad 
    se cumple de forma exacta. En estos casos, uno podría estar tentado a proponer una
    cota más holgada, pero al hacerlo se corre el riesgo de que la nueva cota sea
    aún más difícil de demostrar. Una mejor opción es proponer una nueva cota
    que sea igual a la original pero con la única diferencia de que se le está restando
    el término de menor grado que impide que la inducción converja. 
    
    \paragraph{Nueva H.I.}
    Supóngase que $T(\lfloor n/2\rfloor)\leq c\lfloor n/2\rfloor-d$ 
    y que $T(\lceil n/2\rceil)\leq c\lceil n/2\rceil-d$, donde $d\in\mathbb{N}_{0}$.
    
    \paragraph{Paso inductivo}
    Sustituyendo la hipótesis inductiva en la recurrencia original, se tiene lo sig.:
    \begin{align*}
        T(n) &\leq c\lfloor n/2\rfloor-d+c\lceil n/2\rceil-d+1\\
	    &=cn-2d+1\\
	    &\leq cn-d
    \end{align*}
    Esta nueva cota se cumple para todo valor de $d\geq 1$. Ahora sólo queda demostrar
    que la cota se cumple para la condición de paro de la recurrencia.
    
    \paragraph{Caso base}
    Obsérvese que $n=2$ es el valor más pequeño que resulta en $T(1)$ tanto para 
    $T(\lfloor n/2 \rfloor)$ como para $T(\lceil n/2 \rceil)$. Aplicando la hipótesis
    inductiva a estos valores, se tiene que:
    
    \begin{align*}
        T(2) &\leq 2c-d\\
        T(\lfloor 2/2 \rfloor) + T(\lceil 2/2 \rceil) + 1 &\leq 2c-d \\
        2T(1) + 1 &\leq 2c - d \\
        \dfrac{2T(1)+1+d}{2} &\leq c \\
        T(1) + \dfrac{1+d}{2} &\leq c
    \end{align*}
    \exend
\end{expl}

\section{Cambio de variable}

En ocasiones, se puede aplicar álgebra para transformar una recurrencia
en alguna otra cuya solución ya se conozca, eliminando
la necesidad de recurrir a cualquiera de los métodos anteriores.

\begin{expl}
    Sea $T(n)=2T(\lfloor \sqrt{n} \rfloor)+\lg{n}$. Esta recurrencia se 
    puede simplificar con un cambio de variable. Proponiendo que $m=\lg{n}$,
    se tiene que $T(2^m)=2T(2^{m/2})+m$. Si además se propone una 
    $S(m)=T(2^m)$, se obtiene la recurrencia $S(m)=2S(m/2)+m$, que
    es idéntica a la recurrencia resuelta en el ejemplo \ref{ex1}, por
    lo que la solución es la misma para ambas. Entonces, se tiene que 
    $S(m)=O(m\log m)$. Cambiando de regreso $S(m)$ por $T(n)$, se obtiene
    $T(n)=O(\log{n}\cdot\log\log{n})$.
    \exend
\end{expl}

\section*{Notas bibliográficas}

En las págs. 97-106 del libro de \textcite{cormen_introduction_2009} se
proporciona la demostración del teorema maestro. 
El método de \textcite{drmota_master_2013} es una extensión del método de 
Akra-Bazzi que trabaja con variables discretas. 

Material consultado:
\begin{itemize}
    \item \textcite{cormen_introduction_2009}, págs. 65-67, 83-106 y 112-113.
    \item \textcite{akra_solution_1998}.
    \item \textcite{drmota_master_2013}.
    \item \textcite{goodrich_algorithm_2001}, pág. 12.
\end{itemize}

    
    \appendix
        \chapter{Redacción Matemática}

A diferencia de otras áreas de la ciencia, el estudio de las matemáticas
no se dedica a la recolección de evidencia física, sino a la manipulación
de conceptos abstractos y la presentación de argumentos lógicos. Es
por ello que es de suma importancia desarrollar la habilidad de buena
redacción en esta área (y, por extensión, en las ciencias de la 
computación). A continuación se proporcionan algunos consejos sobre cómo 
redactar correctamente ``trabajos matemáticos''
(i.e. artículos, reportes, ensayos, tareas, etc.).

\section{Organización del contenido}

Antes de comenzar a elaborar un trabajo escrito, es recomendable preparar
lo sig.:

\begin{itemize}
    \item Los antecedentes y la motivación del trabajo.
    \item Las definiciones y la notación a ser utilizadas.
    \item Los ejemplos a ser incluidos.
    \item Los resultados a ser presentados con sus demostraciones correspondientes
    (posiblemente en borrador). 
    \item Referencias a otros resultados que se pretenden utilizar.
    \item El orden en que todo lo anterior será presentado.
\end{itemize}

Si se prepara de antemano esta información, es más fácil escribir
el trabajo, pues sólo se requiere seguir el plan trazado.

\section{Uso apropiado de símbolos}

\begin{itemize}
    \item Nunca se debe iniciar una oración con un símbolo o una ecuación; es 
    preferible iniciar con un sustantivo seguido del símbolo que se pretende utilizar.
    Por ejemplo, en lugar de escribir:
    
    \[
        \pi\text{ es un número irracional.}
    \]
        
    es preferible escribir:
    
    \[
        \text{La constante }\pi\text{ es un número irracional.}
    \]
    
    \item De ser posible, se deben utilizar palabras en lugar de comas para separar
    aquellos símbolos que no conformen una lista. Por ejemplo, en lugar
    de escribir:
    
    \[
        \text{Al igual que }\pi\text{, }e\text{ es un número irracional.}
    \]
        
    es preferible escribir:
    
    \[
        \text{Al igual que }\pi\text{, la constante }e\text{ es un número irracional.}
    \]

    \item Se debe evitar el uso de símbolos lógicos (como $\forall$,
    $\exists$, $\Rightarrow$, i.a.) si el tema que se está tratando no
    pertenece al área de la lógica. Estos símbolos se utilizan comúnmente 
    para sustituir frases matemáticas de uso frecuente. Esta práctica es 
    aceptable si se está trabajando en un borrador o se están tomando apuntes. 
    Sin embargo, cuando se trata de trabajos formales o profesionales, 
    esta práctica debe evitarse.

    \item Se debe evitar el uso de las abreviaciones ``i.e.'' y ``e.g.''
    si se están utilizando símbolos parecidos. Si no se tiene cuidado de cómo 
    se utilizan, pueden hacer confusa la redacción. Por ejemplo, 
    en lugar de escribir ``las expresiones
    $\sqrt{-1}$ y $\lim_{n\to\infty}\left(1+1/n\right)^{n}$ no son números
    racionales; i.e. $i$ y $e$ son irracionales'' es preferible escribir:
    ``las expresiones $\sqrt{-1}$ y $\lim_{n\to\infty}\left(1+1/n\right)^{n}$
    no son números racionales; esto es, las constantes $i$ y $e$ son
    irracionales''.

    \item Todo número utilizado como un cuantificador se debe escribir con letra.
    Por ejemplo, es preferible escribir:
    
    \[
        \text{Hay exactamente dos grupos de orden 4.}
    \]
    \[
        \text{Cincuenta millones de personas no pueden estar equivocadas.}
    \]
    \[
        \text{Hay un millón de enteros positivos menores a 1,000,001.}
    \]
    
    \item No se deben mezclar símbolos con palabras en una misma oración. Por
    ejemplo, en lugar de escribir:
    
    \[
        \text{Todo entero }>1\text{ es primo o compuesto.}
    \]
    
    es preferible escribir:
    
    \[
        \text{Todo entero que mayor a 1 es primo o compuesto.}
    \]

    Otro ejemplo; aunque la sig. oración puede sonar correcta, está mal
    redactada:
    
    \[
        \text{Como }(x-2)(x-3)=0\text{, se tiene que }x=2\text{ o }3.
    \]
    
    En este caso, es preferible escribir:
    
    \[
        \text{Como }(x-2)(x-3)=0\text{, se tiene que }x=2\text{ o }x=3.
    \]

    \item No se deben introducir símbolos si no se van a utilizar. Por ejemplo,
    en la oración ``toda función biyectiva $f$ tiene una inversa'',
    si el símbolo $f$ no se vuelve a utilizar en el texto, entonces
    es preferible omitirlo. 

    \item No se deben usar símbolos sin primero haberlos introducido. Por ejemplo,
    si se tiene la expresión $n=2k+1$ y esta es la primera vez que aparece
    el símbolo $k$, entonces es preferible redactar esta expresión como:
    
    \[
        \text{Sea }k\text{ un entero, se tiene que }n=2k+1.
    \]
    \[
        \text{Se tiene que }n=2k+1\text{, donde }k\text{ es un entero.}
    \]
    
    \item Se deben seguir convenciones bien establecidas sobre el uso de
    algunos símbolos. Hay convenciones que se vienen utilizando en la
    literatura desde hace mucho tiempo y, debido a ello, uno debe apegarse
    a ellas. Algunas de estas convenciones son:

    \begin{itemize}
        \item Las letras $a$, $b$ y $c$ representan constantes. 
        \item Las letras $x$, $y$ y $z$ representan variables. 
        \item Las letras $m$ y $n$ representan enteros. 
        \item Las letras $i$ y $j$ representan índices.
        \item Las letras $f$, $g$ y $h$ representan funciones. 
        \item Las letras $u$, $v$ y $w$ representan vectores. 
        \item Las letras mayúsculas representan conjuntos o matrices.
    \end{itemize}
    
    \item Se deben utilizar parejas apropiadas de símbolos. Por ejemplo, usar
    $a$ junto con $b$ o utilizar $p$ junto con $q$, etc.

\end{itemize}

\section{Escritura de expresiones algebraicas}

Como regla general, si una expresión algebraica es relativamente corta,
entonces se puede escribir en el mismo renglón que el resto del texto.
De lo contrario, es preferible dedicarles su propio renglón, para evitar 
que una expresión larga tenga que separarse en dos renglones. 
Por otro lado, en el caso de manipulaciones algebraicas extensas, es 
recomendable saltar de renglón cada vez que se introduce
un símbolo de comparación. También es recomendable alinear todos los
símbolos de comparación en una misma columna. Por último, en aquellos
casos donde sea inevitable separar una expresión en dos renglones, es 
preferible hacerlo de tal forma que el primer renglón termine con un 
símbolo de operación y el siguiente comience con un término. De esta forma,
queda claro para el lector que la expresión en el primer renglón
está incompleta y continúa en el siguiente. 

\section{Uso apropiado de frases comunes}

\begin{itemize}

    \item Se debe utilizar el artículo ``se'' en lugar de los pronombres ``yo'',
    ``nosotros'' y ``uno''. Usar ``yo'' se considera egocéntrico,
    a menos que se hable sobre alguna experiencia personal. Usar ``nosotros''
    es adecuado, pero se corre el riesgo de sonar demasiado informal. El
    uso de ``uno'' es posible e incluso preferible en algunos casos,
    pero en otros puede ser incorrecto. El uso del artículo ``se'', seguido
    de un verbo conjugado adecuadamente, es una forma de redacción 
    muy flexible y por eso se recomienda su uso sobre los pronombres. 
    Por ejemplo, considere las sig. oraciones y compare cuál suena más apropiado:
    
    \begin{itemize}
        \item Ahora demostraré que $n$ es par.
        \item Ahora demostraremos que $n$ es par.
        \item Ahora uno demostrará que $n$ es par.
        \item Ahora se demostrará que $n$ es par.
    \end{itemize}
    
    \item Se debe evitar el uso de palabras como ``claramente'', ``obviamente'',
    ``trivialmente'', i.a. Al usar estas frases, 
    se corre el riesgo de que lo que se describe a continuación no quede 
    absolutamente claro para el lector, en cuyo caso se dejaría una mala impresión
    sobre el dominio del tema o una mala experiencia de lectura.
    
    \item Se debe tener cuidado de cómo se utilizan las frases ``para cualquier'',
    ``para algún'' y ``para todo''. Estas frases se consideran como
    equivalentes en la redacción matemática pero, si no se tiene
    cuidado del contexto en que se usan, pueden resultar en ambigüedades.
    Por ejemplo, considere la sig. oración: ``Se dice que el conjunto
    $S$ satisface la propiedad $P$ si $P$ se satisface para cualquier
    elemento $s\in S$''. En este caso, queda ambigüo si la propiedad
    $P$ se debe cumplir para todos los elementos de $S$ o solamente
    para uno. En situaciones como esta, es preferible evitar completamente
    la frase ``para cualquier'' en favor de una redacción más explícita;
    e.g. utilizando frases como ``para todo/cada'', ``para algún''
    o ``para al menos uno''.
    
    \item Es incorrecto utilizar la palabra ``como'' en conjunto con ``entonces''.
    Por ejemplo, en lugar de escribir:
    
    \[
        \text{Como }n^{2}\text{ es par, entonces }n\text{ también es par.}
    \]
    
    es preferible escribir cualquiera de estas:

    \[
        \text{Si }n^{2}\text{ es par, entonces }n\text{ también es par.}
    \]
    \[
        \text{Como }n^{2}\text{ es par, se tiene que }n\text{ también es par.}
    \]
    \[
        \text{Como }n^{2}\text{ es par, }n\text{ también es par.}
    \]
    
    \item No se debe utilizar en cantidades abusivas frases como ``por lo tanto'',
    ``lo que implica'', ``como consecuencia'', etc. El propósito de
    estas frases es encadenar varios razonamientos para formar un
    argumento lógico. Utilizar la misma frase de forma repetitiva
    puede volverse muy tedioso para el lector. Debido a ello, se recomienda 
    utilizar una variedad más amplia de frases y, por lo menos, se debe evitar 
    utilizar la misma frase dos veces seguidas. Naturalmente, esto requiere de 
    creatividad y práctica.

\end{itemize}

\section{Consejos generales}

\begin{itemize}
    \item Se deben escribir oraciones completas y que sean gramatical y ortográficamente
    correctas. Se sugiere conjugar igual todos los verbos de una misma oración y utilizar
    conjugaciones simples siempre que sea posible. 
    \item La escritura es un proceso iterativo; rara vez se puede escribir algo
    correctamente en el primer intento. Es por ello que se recomienda
    releer \emph{desde el principio} el trabajo que se está escribiendo
    cada vez que se le hace alguna corrección. También es recomendable
    dejar descansar el escrito unos días antes de volver a revisarlo. 
    \item A la hora de escribir, se debe considerar la audiencia para la que
    se está escribiendo; esto es, se debe elegir el vocabulario y estilo
    de redacción adecuados para ella y se debe considerar su conocimiento
    previo al decidir cuánta información incluir y cuánta omitir. 
    \item Nunca se debe entregar o publicar un trabajo escrito sin antes permitir
    que alguien más lo lea y proporcione retroalimentación. Además, toda
    corrección realizada al escrito debe ser verificada por la misma persona
    que sugirió dicha corrección.
\end{itemize}

\section*{Notas bibliográficas}

\begin{itemize}
    \item \textcite{chartrand_mathematical_2012}, págs. 1-13.
\end{itemize}

        
\chapter{La Potencia y el Logaritmo}

\section{Potencia}
\begin{defn}[Potencia]
Sean $x\in\mathbb{R}$ y $n\in\mathbb{N}_{0}$, donde $x$ se denomina
\emph{base} y $n$ se denomina \emph{exponente}. La $n$-ésima potencia\emph{
}de $x$, denotada como $x^{n}$, es el producto de $n$ bases $x$.
Esto es, $x^{n}=x\cdot x\cdot\dots\text{ (}n\text{ veces)}$.
\end{defn}
\begin{thm}[Leyes de los exponentes]
\label{laws-of-exp}Sean $x,y\in\mathbb{R}$ y sean $m,n\in\mathbb{N}_{0}$.
Se tienen las sig. identidades sobre potencias:
\begin{enumerate}
\item $x^{0}\triangleq1$
\item $x^{1}\triangleq x$
\item $x^{-1}\triangleq1/x$
\item $x^{m/n}\triangleq\sqrt[n]{x^{m}}$
\item $x^{m}x^{n}=x^{m+n}$
\item $x^{m}/x^{n}=x^{m-n}$, donde $x\neq0$
\item $(x^{m})^{n}=x^{mn}$
\item $(xy)^{n}=x^{n}y^{n}$
\item $(x/y)^{n}=x^{n}/y^{n}$, donde $y\neq0$
\end{enumerate}
\end{thm}
\begin{rem}
Cuando $x=0$, se tiene que $x^{-1}$ es indefinido.
\end{rem}
%
\begin{rem}
Cuando $x=0$, por convención, la identidad $x^{0}=1$ se seguirá
cumpliendo.
\end{rem}
\begin{proof}[Demostración para el teorema \ref{laws-of-exp}]
Las primeras cuatro identidades se aceptan sin requerir una demostración.
El resto se pueden demostrar de forma algebraica. Comenzando por la
identidad 5, ésta se obtiene de:

\[
x^{m}x^{n}=x\cdot x\cdot\dots\text{ (}m\text{ veces)}\cdot x\cdot x\cdot\dots\text{ (}n\text{ veces)}=x^{m+n}
\]
Para la identidad 6 se tienen dos casos:
\begin{casenv}
\item Si $m<n$, se tiene que
\begin{align*}
\dfrac{x^{m}}{x^{n}} & =\dfrac{x\cdot x\cdot\dots\text{ (}m\text{ veces)}}{x\cdot x\cdot\dots\text{ (}n\text{ veces)}}\\
 & =1\cdot1\cdot\dots\text{ (}m\text{ veces)}\cdot\dfrac{1}{x}\cdot\dfrac{1}{x}\cdot\dots\text{ (}n-m\text{ veces)}\\
 & =x^{-1}\cdot x^{-1}\cdot\dots\text{ (}n-m\text{ veces)}\\
 & =x^{m-n}
\end{align*}
\item Si $m\geq n$, se tiene que
\begin{align*}
\dfrac{x^{m}}{x^{n}} & =\dfrac{x\cdot x\cdot\dots\text{ (}m\text{ veces)}}{x\cdot x\cdot\dots\text{ (}n\text{ veces)}}\\
 & =1\cdot1\cdot\dots\text{ (}n\text{ veces)}\cdot x\cdot x\cdot\dots\text{ (}m-n\text{ veces)}\\
 & =x^{m-n}
\end{align*}
\end{casenv}
La identidad 7 se obtiene de:

\[
\left(x^{m}\right)^{n}=\left[x\cdot x\cdot\dots\text{ (}m\text{ veces)}\right]\cdot\left[x\cdot x\cdot\dots\text{ (}m\text{ veces)}\right]\cdot\dots\text{ (}n\text{ veces)}=x^{mn}
\]
La identidad 8 se obtiene de:

\begin{align*}
\left(xy\right)^{n} & =xy\cdot xy\cdot\dots\text{ (}n\text{ veces)}\\
 & =x\cdot x\cdot\dots\text{ (}n\text{ veces)}\cdot y\cdot y\cdot\dots\text{ (}n\text{ veces)}\\
 & =x^{n}y^{n}
\end{align*}
Finalmente, la identidad 9 se obtiene de:

\begin{align*}
\left(\dfrac{x}{y}\right)^{n} & =\dfrac{x}{y}\cdot\dfrac{x}{y}\cdot\dots\text{ (}n\text{ veces)}\\
 & =\dfrac{x\cdot x\cdot\dots\text{ (}n\text{ veces)}}{y\cdot y\cdot\dots\text{ (}n\text{ veces)}}\\
 & =\dfrac{x^{n}}{y^{n}}
\end{align*}
\end{proof}

\section{Logaritmo}
\begin{defn}[Logaritmo]
Sean $b,x\in\mathbb{R}^{+}$, donde $b>1$ y se denomina \emph{base}.
El logaritmo de \emph{$x$} con respecto a \emph{$b$, }denotado como
$\log_{b}x$, es la potencia a la que hay que elevar $b$ para obtener
$x$ como resultado. Esto es, $y=\log_{b}x$ si y sólo si $b^{y}=x$,
donde $y\in\mathbb{R}$.
\end{defn}
%
\begin{defn}[Logaritmo decimal]
Se dice que un logaritmo es decimal cuando su base es 10. El logaritmo
decimal de un número real $x$ se denota como $\log x$. 
\end{defn}
%
\begin{defn}[Logaritmo natural]
Se dice que un logaritmo es natural cuando su base es la constante
de Neper $e$. El logaritmo natural de un número real $x$ se denota
como $\ln x$.
\end{defn}
%
\begin{defn}[Logaritmo binario]
Se dice que un logaritmo es binario cuando su base es 2. El logaritmo
binario de un número real $x$ se denotan como $\lg x$.
\end{defn}
\begin{prop}
\label{no-negative-bases}La base de un logaritmo no puede ser negativa.
\end{prop}
\begin{proof}
Supóngase que $y=\log_{b}x$, donde $b$ es negativo. Esto implica
que $b^{y}=x$ (por la definición del logaritmo). Sin embargo, por
las leyes de los signos, se tiene que el término $b^{y}$ es positivo
cuando $y$ es par y negativo cuando $y$ es impar. Esto implica que
existen valores de $x$ para los cuales $\log_{b}x$ no existe.
\end{proof}
\begin{prop}
\label{no-negative-logarithm}Los números negativos no tienen logaritmo.
\end{prop}
\begin{proof}
Supóngase que $y=\log_{b}x$, donde $x$ es negativo. Esto implica
que $b^{y}=x$ (por la definición del logaritmo). Sin embargo, como
consecuencia de la proposición \ref{no-negative-bases}, se tiene
que $b$ debe ser positiva. Como la potencia de todo número positivo
siempre resulta en un número positivo (por las leyes de los signos),
entonces no existe un número $y$ que satisfaga la relación $b^{y}=x$.
\end{proof}
\begin{prop}
Se tiene que $\log_{b}1=0$ para toda $b>1$.
\end{prop}
\begin{proof}
Supóngase que $y=\log_{b}1$, lo que implica que $b^{y}=1$ (por la
definición del logaritmo). Se tiene entonces lo sig.:

\begin{align*}
b^{y} & =1\\
b^{y} & =b^{0}\\
y & =0\\
\log_{b}1 & =0
\end{align*}
\end{proof}
\begin{prop}
Se tiene que $\log_{b}b=1$ para toda $b>1$.
\end{prop}
\begin{proof}
Supóngase que $y=\log_{b}b$, lo que implica que $b^{y}=b$ (por la
definición del logaritmo). Se tiene entonces lo sig.:

\begin{align*}
b^{y} & =b\\
b^{y} & =b^{1}\\
y & =1\\
\log_{b}b & =1
\end{align*}
\end{proof}
\begin{prop}
El logaritmo de todo número mayor a 1 es positivo. Esto es, $\log_{b}x>0$
para toda $x>1$.
\end{prop}
\begin{proof}
Supóngase que $y=\log_{b}x$, donde $x>1$. Esto implica que $b^{y}=x$
(por la definición del logaritmo). De la proposición (\ref{no-negative-bases})
se tiene que $b$ es positivo. Por lo tanto, para que se satisfaga
la relación $b^{y}=x$, el valor de $y$ debe ser positivo. 
\end{proof}
%
\begin{prop}
El logaritmo de todo número menor a 1 es negativo. Esto es, $\log_{b}x<0$
para toda $x\in(0,1)$.
\end{prop}
\begin{proof}
Supóngase que $y=\log_{b}x$, donde $x\in(0,1)$. Esto implica que
$b^{y}=x$ (por la definición del logaritmo). De las proposiciones
(\ref{no-negative-bases}) y (\ref{no-negative-logarithm}) se tiene
que $b$ y $x$ son ambos positivos. Por lo tanto, para que se satisfaga
la relación $b^{y}=x$, el valor de $y$ debe ser negativo (por las
leyes de los exponentes).
\end{proof}
\begin{thm}[Leyes de los logaritmos]
Sean $b,v,x,y\in\mathbb{R^{+}}$, donde $b,v>1$. Se tienen las sig.
identidades sobre logaritmos:
\begin{enumerate}
\item $\log_{b}\left(x\cdot y\right)=\log_{b}x+\log_{b}y$
\item $\log_{b}\left(x/y\right)=\log_{b}x-\log_{b}y$
\item $\log_{b}x^{y}=y\cdot\log_{b}x$
\item $\log_{b}\sqrt[y]{x}=1/y\cdot\log_{b}x$
\item $\log_{b}x=\log_{v}x/\log_{v}b$
\item $\log_{b}x=1/\log_{x}b$
\item $x^{\log_{b}y}=y^{\log_{b}x}$
\end{enumerate}
\end{thm}
\begin{proof}
Todas las identidades se pueden demostrar algebraicamente. Comenzando
por la identidad 1, sean $\alpha,\beta\in\mathbb{R}$ tales que $\alpha=b^{x}$
y $\beta=b^{y}$. Entonces, se tiene que:

\begin{align*}
\alpha\cdot\beta & =b^{x}b^{y}\\
\alpha\cdot\beta & =b^{x+y}\\
\log_{b}(\alpha\cdot\beta) & =x+y\\
\log_{b}(\alpha\cdot\beta) & =\log_{b}\alpha+\log_{b}\beta
\end{align*}
La identidad 2 se origina de una manipulación algebraica similar:

\begin{align*}
\dfrac{\alpha}{\beta} & =\dfrac{b^{x}}{b^{y}}\\
\dfrac{\alpha}{\beta} & =b^{x-y}\\
\log_{b}\left(\dfrac{\alpha}{\beta}\right) & =x-y\\
\log_{b}\left(\dfrac{\alpha}{\beta}\right) & =\log_{b}\alpha-\log_{b}\beta
\end{align*}
Para la identidad 3, sea $\theta\in\mathbb{R}$ tal que $\theta=\log_{b}x$.
Esto implica que $b^{\theta}=x$ (por la definición del logaritmo),
por lo que se tiene lo sig.:

\begin{align*}
b^{\theta} & =x\\
(b^{\theta})^{y} & =x^{y}\\
b^{\theta y} & =x^{y}\\
\theta\cdot y & =\log_{b}x^{y}\\
y\cdot\log_{b}x & =\log_{b}x^{y}
\end{align*}
La identidad 4 se obtiene de:

\begin{align*}
b^{\theta} & =x\\
\sqrt[y]{b^{\theta}} & =\sqrt[y]{x}\\
b^{\theta/y} & =\sqrt[y]{x}\\
\dfrac{\theta}{y} & =\log_{b}\sqrt[y]{x}\\
\dfrac{1}{y}\cdot\log_{b}x & =\log_{b}\sqrt[y]{x}
\end{align*}
La identidad 5 se obtiene de:

\begin{align*}
b^{\theta} & =x\\
\log_{v}b^{\theta} & =\log_{v}x\\
\theta\cdot\log_{v}b & =\log_{v}x\\
\theta & =\dfrac{\log_{v}x}{\log_{v}b}\\
\log_{b}x & =\dfrac{\log_{v}x}{\log_{v}b}
\end{align*}
La identidad 6 se obtiene de:

\begin{align*}
b^{\theta} & =x\\
\log_{x}b^{\theta} & =1\\
\theta\cdot\log_{x}b & =1\\
\theta & =\dfrac{1}{\log_{x}b}\\
\log_{b}x & =\dfrac{1}{\log_{x}b}
\end{align*}
Finalmente, para la identidad 7, supóngase que $\alpha=\log_{b}x$
y $\beta=\log_{b}y$. Esto implica que $b^{\alpha}=x$ y que $b^{\beta}=y$
(por la definición del logaritmo). Entonces, se tiene que:

\begin{align*}
x^{\log_{b}y} & =x^{\beta}\\
 & =(b^{\alpha})^{\beta}\\
 & =b^{\alpha\beta}\\
 & =(b^{\beta})^{\alpha}\\
 & =y^{\alpha}\\
 & =y^{\log_{b}x}
\end{align*}
\end{proof}
\begin{rem}
Debido a que cambiar la base de cualquier logaritmo altera dicho valor
en un factor constante, el comportamiento asintótico sigue siendo
el mismo sin importar la base. Como consecuencia, la base del logaritmo
suele omitirse al usar notación asintótica.
\end{rem}


    
    \printbibliography[heading=bibintoc]
\end{document}

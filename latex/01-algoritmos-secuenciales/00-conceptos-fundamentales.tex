%% LyX 2.3.5.2 created this file.  For more info, see http://www.lyx.org/.
%% Do not edit unless you really know what you are doing.
\documentclass[letterpaper,oneside,spanish]{book}
\PassOptionsToPackage{vlined,linesnumbered,ruled}{algorithm2e}
\usepackage[T1]{fontenc}
\usepackage[utf8]{inputenc}
\setcounter{secnumdepth}{3}
\usepackage{pifont}
\usepackage{textcomp}
\usepackage{algorithm2e}
\usepackage{amsmath}
\usepackage{amsthm}

\makeatletter

%%%%%%%%%%%%%%%%%%%%%%%%%%%%%% LyX specific LaTeX commands.
\pdfpageheight\paperheight
\pdfpagewidth\paperwidth


%%%%%%%%%%%%%%%%%%%%%%%%%%%%%% User specified LaTeX commands.
\usepackage{indentfirst}

\usepackage{titlesec}
\titleformat{\chapter}[display]{\bfseries\centering\larger}{\normalsize{\chaptertitlename\ \thechapter}}{0.5ex}{}[]
\titlespacing*{\chapter}{0pt}{-50pt}{20pt}
\titleformat{\section}[hang]{\bfseries\normalsize}{\thesection\ }{0pt}{}[]
\titleformat{\subsection}[runin]{\bfseries\normalsize}{\thesubsection\ }{0pt}{}[]
\titleformat{\paragraph}[runin]{\itshape\normalsize}{\theparagraph\ }{0pt}{}[]

\newtheoremstyle{myplain}{\topsep}{\topsep}{\itshape}{}{\scshape}{.}{.5em}{}
\newtheoremstyle{mydef}{\topsep}{\topsep}{\normalfont}{}{\scshape}{.}{.5em}{}
\theoremstyle{myplain}
\newtheorem{mythm}{Teorema}[]
\newtheorem{mylem}{Lema}[]
\newtheorem{myprop}{Proposición}[]
\theoremstyle{mydef}
\newtheorem{mydef}{Definición}[]
\newtheorem{myrmk}{Observación}[]
\newtheorem{myex}{Ejemplo}[]

\let\thm\mythm
\let\endthm\endmythm
\let\lem\mylem
\let\endlem\endmylem
\let\prop\myprop
\let\endprop\endmyprop
\let\defn\mydef
\let\enddefn\endmydef
\let\example\myex
\let\endexample\endmyex
\let\rem\myrmk
\let\endrem\endmyrmk

\usepackage{xpatch}
\newcommand{\proofnamefont}{\scshape}
\xpatchcmd{\proof}{\itshape}{\normalfont\proofnamefont}{}{}

\AtBeginDocument{
  \def\labelitemi{\Pisymbol{psy}{183}}
  \def\labelitemii{\Pisymbol{psy}{45}}
  \def\labelitemiii{\Pisymbol{psy}{215}}
}

\makeatother

\usepackage{babel}
\addto\shorthandsspanish{\spanishdeactivate{~<>.}}

\begin{document}

\chapter*{Conceptos Fundamentales}

\section*{Problema computacional}

Un problema computacional (o, simplemente, \emph{problema}) es una
relación explícita entre dos conjuntos de entidades, uno de \emph{entrada}
y otro de \emph{salida}. Por ejemplo, el problema de \textsc{Ordenamiento}
se define de la sig. manera:
\begin{itemize}
\item \emph{Entrada}: una secuencia de $n$ elementos comparables, $A=\langle a_{1},a_{2},\dots,a_{n}\rangle$.
\item \emph{Salida}: una permutación de la secuencia de entrada, $A\text{´}=\langle a\text{´}_{1},a\text{´}_{2},\dots,a\text{´}_{n}\rangle$,
tal que $a\text{´}_{1}\leq a\text{´}_{2}\le\dots\le a\text{´}_{n}$.
\end{itemize}

\section*{Algoritmo}

Un algoritmo es una secuencia finita y ordenada de instrucciones bien
definidas que describen el procedimiento a seguir para resolver un
problema determinado; i.e. para transformar la entrada de dicho problema
a la salida correspondiente. Se dice que un algoritmo es
\begin{itemize}
\item \emph{finito} porque tiene una cantidad determinada de instrucciones
y su ejecución termina en una cantidad determinada de tiempo,
\item \emph{ordenado} porque, dadas dos instrucciones cualesquiera, no hay
ambigüedad sobre cuál debe ejecutarse primero,
\item \emph{bien definido} porque no hay ambigüedad sobre la interpretación
de las instrucciones.
\end{itemize}
Un \emph{caso específico} (instance) para algún problema particular
es cualquier conjunto de valores específicos que satisfagan los requerimientos
de entrada impuestos por dicho problema. Por ejemplo, dos casos específicos
para el problema de ordenamiento son: $\langle31,41,59,26,41,58\rangle$
y $\langle f,x,a,b,w,j\rangle$.

Se dice que un algoritmo es \emph{correcto} cuando, para cualquier
caso específico, el algoritmo termina su ejecución y produce un resultado
que cumple los requerimientos de salida impuestos por el problema.
Se dice que un algoritmo correcto \emph{resuelve} el problema en cuestión.
Se dice que un algoritmo es \emph{incorrecto} cuando existe al menos
un caso específico para el cual el algoritmo no termina su ejecución
o termina con un resultado que no cumple con los requerimientos de
salida. 

\section*{Estructura de datos}

Una estructura de datos es una forma de organizar, almacenar y manipular
un conjunto de datos de tal forma que se facilite su acceso y modificación.
En otras palabras, es una colección de datos y de las operaciones
que pueden aplicarse sobre ellos.

\section*{Pseudocódigo}

El pseudocódigo es una forma de escribir un algoritmo. Consiste de
una descripción compacta, informal y de alto nivel que utiliza estructuras
de control de flujo (i.e. sentencias condicionales, bucles, i.a.)
combinadas con prosa en lenguaje natural y/o notación matemática.
El propósito del pseudocódigo es que el algoritmo que describe sea
fácil de leer y comprender para una persona. A diferencia de un lenguaje
de programación, el pseudocódigo no puede ser leído ni ejecutado por
una computadora. Se dice que el pseudocódigo es de alto nivel porque
omite muchos detalles sobre la implementación del proceso que describe.

\section*{Análisis de algoritmos}

El análisis de algoritmos es el conjunto de técnicas matemáticas por
medio de las cuáles se pueden caracterizar las propiedades particulares
de un algoritmo de forma independiente a su implementación específica
en hardware y/o software. En estos apuntes, el análisis consistirá
en describir únicamente dos características:
\begin{itemize}
\item \emph{La exactitud} (correctness): ¿el algoritmo es correcto?
\item \emph{La eficiencia}: ¿qué tanto crece el tiempo de ejecución requerido
por el algoritmo con respecto al tamaño de la entrada? ¿dicha razón
de crecimiento es polinomial?
\end{itemize}

\section*{Notas bibliográficas}

Realmente no existe una definición formal de lo que es un algoritmo;
la ``definición'' que se presenta aquí es una lista de las características
que posee un algoritmo, más que una definición propia.
\begin{itemize}
\item Cormen T.H., Leiserson C.E., Rivest R.L. \& Stein C., ``Introduction
to Algorithms'', 3ra ed. (2009), MIT Press. Págs. 5-14 y 20-22.
\item Skiena S.S., ``The Algorithm Design Manual'', 2da ed. (2012), Springer.
Págs. 3-13.
\end{itemize}

\end{document}

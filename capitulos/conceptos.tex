\chapter{Conceptos fundamentales}

Un \textbf{problema computacional} es una relación entre dos valores o conjuntos de valores: uno de \textbf{entrada} y otro de \textbf{salida}. 
Por ej., el problema de \textsc{Ordenamiento} se define formalmente de la sig. manera:
\begin{itemize}
  \item \emph{Entrada}: una secuencia de \(n\in\mathbb{N}\) elementos comparables, \(A=\{a_1,a_2,\dots,a_n\}\).
  \item \emph{Salida}: una permutación de la secuencia de entrada, \(A'=\{a'_1,a'_2,\dots,a'_n\}\), tal que \(a'_1\leq a'_2\leq\dots\leq a'_n\).
\end{itemize}

Un \textbf{caso específico} para un problema computacional determinado es cualquier valor o conjunto de valores que satisfacen la descripción de la entrada del problema.
Por ej., dos casos específicos para el problema de \textsc{Ordenamiento} son las secuencias \(\{4,7,5,1\}\) y \(\{d,x,j,e\}\).
% Los casos específicos de un algoritmo determinado pueden categorizarse en varias \textbf{clases de entrada} que representan todas las diferentes formas en que puede presentarse la entrada de dicho algoritmo.
% P. ej., un algoritmo que admite como entrada dos números, \(a\) y \(b\), tendría tres clases de entrada: \(a<b\), \(a=b\) y \(a>b\).

Un \textbf{algoritmo} es una secuencia de instrucciones inambiguas para transformar la entrada de un problema computacional determinado a la salida correspondiente.
Un algoritmo es \textbf{correcto} si y solo si, para cada caso específico, el algoritmo termina su ejecución y produce un resultado que cumple la descripción de la salida del problema.
Se dice que un algoritmo \textbf{resuelve} el problema en cuestión si y solo si es correcto.

Una \textbf{estructura de datos} es una colección de reglas y procedimientos para organizar, accesar y manipular un conjunto de datos.

\marginnote[1.0\baselineskip]{%
  Dependiendo de la aplicación y el tipo de algoritmo que se está analizando, existen muchas otras características que también podrían ser de interés analizarlas.
  Por ej., una característica frecuentemente estudiada es el orden de crecimiento de la cantidad de memoria requerida por el algoritmo con respecto al tamaño de la entrada.
}
El \textbf{análisis de algoritmos} es la colección de técnicas matemáticas que se usan para caracterizar las propiedades particulares de un algoritmo determinado de forma independiente a su implementación en hardware y/o software. 
El análisis de un algoritmo consiste en determinar al menos dos características:
\begin{itemize}
  \item La \textbf{corrección}: ¿el algoritmo es correcto?
  \item La \textbf{eficiencia}: ¿cuál es el \emph{orden de crecimiento} del 
  \emph{tiempo de ejecución} del algoritmo con respecto al \emph{tamaño de su entrada}?
\end{itemize}
Un algoritmo es \textbf{eficiente} si y solo si su orden de crecimiento es polinomial.

\marginnote[-1\baselineskip]{\textbf{Literatura consultada}: \textcite{cormen_2009}, pp. 5-14 y 20-22; \textcite{skiena_2012}, pp. 3-13.}

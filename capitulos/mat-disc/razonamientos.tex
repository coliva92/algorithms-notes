\chapter{Razonamientos lógicos}

Un \textbf{razonamiento lógico} es una secuencia de \(n\) proposiciones, \(P_1,P_2,\dots,P_n\), donde las primeras \(n-1\) proposiciones se denominan \textbf{premisas} y la última se denomina \textbf{conclusión}.
Los razonamientos se escriben con el sig. formato:

1.1

Un razonamiento es \textbf{válido} cuando la conclusión es verdadera si las premisas también lo son.
Para demostrar que un razonamiento determinado es válido:
\begin{itemize}
  \item Demostrar que \((P_1\land P_2\land\dots\land P_{n-1})\Rightarrow P_n\).
  \item Partiendo de las premisas, usar las \emph{reglas de inferencia} para llegar hasta la conclusión.
\end{itemize}

Una \textbf{regla de inferencia} es un razonamiento válido que es particularmente simple y recurrente, por lo que es útil para demostrar la validez de otros razonamientos.
\marginnote{
  ¿Cuál es la diferencia entre la implicación lógica y una regla de inferencia?
}

Para demostrar que un razonamiento determinado inválido:
\begin{itemize}
  \item Construir la tabla de verdad de las premisas y la conclusión y verificar que hay al menos un reglón donde todas las premisas son verdaderas y la conclusión es falsa.
  \item Suponer que la conclusión es falsa y describir con prosa cómo eso conlleva a que todas las premisas sean verdaderas.
\end{itemize}


\newpage
\section{Reglas de inferencia básicas}

\marginnote{%
  Muchas de las reglas de inferencias mostradas aquí aparecen con diferentes nombres en la literatura.
}

1.2

\chapter{Análisis de la corrección de un algoritmo}

El primer paso al analizar un algoritmo es determinar si este es correcto o incorrecto. 
El procedimiento que se sigue para llevar a cabo esta tarea se denomina \textbf{análisis de la corrección de un algoritmo}.

\section{El contraejemplo}

\marginnote[0.5\baselineskip]{Se recomienda buscar un contraejemplo para un algoritmo antes de hacer la demostración de su corrección.}

\marginnote[1.5\baselineskip]{Si no se encuentra un contraejemplo para un algoritmo determinado, esto no implica que dicho algoritmo es correcto.}

\marginnote[1.5\baselineskip]{La demostración de la corrección de un algoritmo debe explicar no solo por qué el algoritmo es correcto, sino también por qué no es incorrecto.}

Para demostrar que un algoritmo determinado es incorrecto, basta con producir un \textbf{contraejemplo}; esto es, un caso específico para el cual el algoritmo no termina su ejecución o no produce un resultado que cumple las características de salida del problema. 
Para encontrar un contraejemplo, se recomienda experimentar primero con los casos específicos más complicados para el algoritmo; p. ej., aquellos que contengan valores empatados o que mezclen valores de frontera de extremos opuestos. 
El contraejemplo debe ser lo más simple y pequeño posible; idealmente la razón de por qué el algoritmo es incorrecto debe ser inmediatamente clara. 
Así, una vez encontrado un contraejemplo, se recomienda simplificarlo tanto como sea posible y presentarlo junto con el resultado incorrecto que produjo el algoritmo y el resultado correcto que debió producir.

\section{La invariante de lazo}

\marginnote[0.5\baselineskip]{%
  La diferencia entre la invariante de lazo e inducción es que el primer método es una aplicación especializada del segundo.
}

Una \textbf{invariante de lazo} es una proposición lógica que se cumple inmediatamente antes de cada iteración y al finalizar la ejecución de un bucle determinado.
La invariante se define en función de cada iteración \(i\).
Demostrar que una invariante de lazo se cumple para todas las iteraciones es un procedimiento análogo a una demostración por inducción y consta de los sig. pasos: 
\begin{enumerate}
  \item \textbf{Inicialización}: se determina el estado en que se encuentra el algoritmo antes de ejecutar la primera iteración y se demuestra que la invariante se cumple bajo estas condiciones.
  \item \textbf{Mantenimiento}: se supone que la invariante se cumple antes de comenzar alguna iteración arbitraria \(j\) y se determina el estado en que se encuentra el algoritmo como consecuencia de ello. 
  Después, se ejecuta la iteración \(j\) y se demuestra que la invariante se cumple antes de comenzar la iteración \(j+1\).
  \item \textbf{Finalización}: se determina cuál es el estado del algoritmo al salir del bucle y se utiliza junto con la invariante para demostrar que el algoritmo es correcto o para caracterizar alguna propiedad particular del mismo. 
\end{enumerate}

\marginnote[-3.8\baselineskip]{%
  Estrictamente hablando, en la finalización se debería demostrar que la invariante se sigue cumpliendo cuando la condición de paro del bucle es falsa.
  Sin embargo, obsérvese que en el mantenimiento se demuestra no solo que la invariante se cumple inmediatamente antes de cada iteración, sino que también se sigue cumpliendo inmediatamente \emph{después} de cada iteración, incluyendo la \emph{última}.
  Es por eso que en la finalización se puede suponer que la invariante se sigue cumpliendo al terminar la ejecución del bucle.
}

\section{Recomendaciones para el diseño de algoritmos correctos}

\paragraph*{Definir la salida del problema correctamente.}{%
  Las características de salida no deben ser ambiguas y deben describir cómo determinar de forma inequívoca si el resultado es correcto.
  Tampoco deben constar de objetivos compuestos, esto es, de múltiples objetivos que deben alcanzarse al mismo tiempo.
}

\paragraph*{Típicamente, entre más restrictivas sean las características de entrada, menor es la dificultad para resolver el problema.}{
  Se recomienda comenzar por diseñar un algoritmo correcto que admita entradas con características muy particulares y después extenderlo a entradas más generales. 
}

\paragraph*{Analizar la complejidad computacional del problema al mismo tiempo que se diseña un algoritmo para resolverlo.}{
  De esta forma, cualquier descubrimiento o avance que se obtenga en uno podría aprovecharse después para avanzar en el otro.
  Además, al final se obtendría uno de dos posibles resultados: ya sea se llega a la construcción de un algoritmo correcto y eficiente o se caracteriza la complejidad computacional del problema.
}

\paragraph*{Modelar el problema de forma adecuada}{
  Las entidades y sus respectivas interacciones que constituyen el problema a resolver en la vida real deben describirse en términos de alguna \emph{estructura abstracta} ya conocida (como son las cadenas, los grafos, los conjuntos, entre otros) y sus operaciones correspondientes.
  Sin embargo, las características de estas entidades reales no siempre se alinean perfectamente con las de la estructura elegida para representarlas.
  En estos casos, se recomienda ignorar temporalmente aquellos detalles que no encajen y decidir más adelante, después de trabajar un tiempo con esa estructura, si dichos detalles son realmente esenciales o no para resolver el problema.
}

\marginnote[-2\baselineskip]{\textbf{Literatura consultada}: \textcite{cormen_2009}, pp. 18-20; \textcite{skiena_2012}, pp. 11-16}

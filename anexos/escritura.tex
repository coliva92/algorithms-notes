\chapter{Redacción matemática}

A continuación se proporcionan algunos consejos sobre cómo redactar apropiadamente cualquier tipo de ``texto matemático'', i.e., tareas, ensayos, reportes, artículos, tesis, etc. que contengan manipulaciones algebraicas y, sobre todo, demostraciones.


\section{Organización del contenido}

Antes de comenzar a escribir cualquier texto matemático, se debe preparar la sig. información:

\begin{itemize}
  \item Los antecedentes y la motivación del trabajo.
  \item Las definiciones y la notación que se utilizarán
  \item Los ejemplos a ser incluidos.
  \item Los resultados a ser presentados (con sus respectivas demostraciones, incluso si están en forma de borrador).
  \item Referencias a otros resultados que se pretenden utilizar.
  \item El orden en que se presentará todo lo anterior.
\end{itemize}


\section{Uso apropiado de símbolos}

Nunca iniciar una oración con un símbolo o una ecuación. 
Es preferible iniciar con un sustantivo seguido del símbolo que se pretende utilizar.
P. ej., en lugar de escribir:
\[
  \text{``}\pi\text{ es un número irracional'',}
\]
es preferible escribir:
\[
  \text{``La constante }\pi\text{ es un número irracional''.}
\]

Se deben usar palabras, en lugar de comas, para separar aquellos símbolos que no conformen una lista o secuencia.
P. ej., en lugar de escribir:
\[
  \text{``Al igual que }\pi\text{, }e\text{ es un número irracional'',}
\]
es preferible escribir:
\[
  \text{``Al igual que }\pi\text{, la constante }e\text{ es un número irracional''.}
\]

Evitar el uso de símbolos lógicos (como \(\forall\), \(\exists\), \(\rightarrow\), i.a.) si en el texto no se está manejando la lógica. 
Una práctica común es utilizar estos símbolos para abreviar frases de uso frecuente en matemáticas.
Esta práctica es aceptable si se trata de textos informales, pero debe evitarse por completo en textos profesionales o académicos.

Evitar abreviaciones como ``i.e.'' o ``e.g.'' si se están utilizando símbolos parecidos.
Si no se tiene cuidado al usar estas abreviaciones, esto puede resultar en una redacción confusa.
P. ej., en lugar de escribir: ``las expresiones
\(\sqrt{-1}\) y \(\lim_{n\to\infty}(1+1/n)^{n}\) no son números
racionales; i.e., \(i\) y \(e\) son irracionales'', es preferible escribir: ``las expresiones \(\sqrt{-1}\) y \(\lim_{n\to\infty}\left(1+1/n\right)^{n}\)
no son números racionales; esto es, las constante $i$ y la constante $e$ son
irracionales''.

Los números usados como adjetivos deben escribirse con letras (si esto no resulta en un texto más intrincado) y aquellos usados para expresar valores concretos de algo deben escribirse con números.
P. ej., en preferible escribir:
\[
  \text{``Hay exactamente dos grupos de orden 4'',}
\]
\[
  \text{``Cincuenta millones de personas no pueden estar equivocadas'',}
\]
\[
  \text{``Hay un millón de enteros positivos menores a 1,000,001''.}
\]

No mezclar símbolos con palabras en forma inapropiada en una oración.
P. ej., en lugar de escribir:
\[
  \text{``Todo entero }>1\text{ es primo o compuesto'',}
\]
es preferible escribir:
\[
  \text{``Todo entero mayor a 1 es primo o compuesto''.}
\]
Otro ejemplo; aunque la sig. oración puede sonar correcta, está mal
redactada:
\[
  \text{``Como }(x-2)(x-3)=0\text{, se tiene que }x=2\text{ o }3\text{'',}.
\]
En este caso, es preferible escribir:
\[
  \text{``Como }(x-2)(x-3)=0\text{, se tiene que }x=2\text{ o }x=3\text{''}.
\]

No introducir símbolos si estos no se utilizan después.
P. ej., en la oración ``toda función biyectiva \(f\) tiene una inversa'',
si el símbolo \(f\) no vuelve a utilizarse en el texto, entonces
debe omitirse.

No usar símbolos sin haberlos introducido.
P. ej., si se tiene la expresión \(n=2k+1\) y esta es la primera vez que aparece
el símbolo \(k\), entonces es preferible redactar esta expresión como
\[
  \text{``Sea }k\text{ un entero, se tiene que }n=2k+1\text{''}.
\]
\[
  \text{``Se tiene que }n=2k+1\text{, donde }k\text{ es un entero''.}
\]

Apegarse de forma consistente a las convenciones tradicionales sobre el uso de símbolos.
P. ej., si \(m\) y \(n\) generalmente se usan para representar enteros, entonces no deben usarse para representar números reales, o si \(f\) suele representar una función, entonces no debe usarse para representar números.
Otro ejemplo, si \(a\) suele usarse junto con \(b\) o si \(x\) suele usarse junto con \(y\), entonces no debe usarse \(a\) junto con \(y\) ni tampoco deben usarse otras combinaciones similares.
El caso es que los símbolos deben usarse de la forma en la que el lector está esperando que se usen, de lo contrario el texto podría confundir al lector.


\section{Escritura de expresiones algebraicas}

Si una expresión algebraica es relativamente corta, entonces puede escribirse en el mismo renglón que el resto del texto.
De lo contrario, es preferible dedicarles su propio renglón, para evitar que una expresión larga tenga que separarse en varios renglones. 

En el caso de manipulaciones algebraicas extensas, se debe saltar de renglón cada vez que se introduce un símbolo de comparación. 
También se deben alinear todos los símbolos de comparación en una misma columna. 

Cuando una expresión es tan larga que es inevitable interrumpirla a medio camino para escribirla en el siguiente renglón, esto se debe hacer de tal forma que el renglón anterior termine con un símbolo de operación o comparación y el siguiente comience con un número o una variable.
De esta forma, queda indicado más claramente que la expresión en el renglón anterior está incompleta y continúa en el siguiente. 


\section{Uso apropiado de frases comunes}

Debe utilizarse el artículo ``se'' en lugar de los pronombres ``yo'', ``nosotros'' y ``uno''.
No se debe usar ``yo'' a menos que se esté describiendo una experiencia personal.
El uso de ``uno'' no siempre es gramaticalmente correcto y podría llevar a confusiones.
Usar ``nosotros'' no es inapropiado pero podría considerarse demasiado inclusivo e informal.
El uso del artículo ``se'', seguido de un verbo conjugado adecuadamente, no sufre de ninguno de estos problemas, además de que es una forma de redacción muy flexible.
Por eso que se recomienda su uso en lugar de usar pronombres.
P. ej., considere las sig. oraciones y compare compare cuál suena más formal y apropiado:
\[
  \text{``Ahora demostraré que }n\text{ es par.''}
\]
\[
  \text{``Ahora demostremos que }n\text{ es par.''}
\]
\[
  \text{``Ahora, uno demostrará que }n\text{ es par.''}
\]
\[
  \text{``Ahora se demostrará que }n\text{ es par.''}
\]

No se deben usar palabras como ``obviamente'', ``evidentemente'', ``ciertamente'', etc. pues siempre existe la posibilidad de que el lector no comprende lo que se describe inmediatamente después de estas palabras, ya sea porque la redacción no es suficientemente clara o porque el lector carece de algún conocimiento requerido.
En cualquier caso, si el lector no comprende algo que el autor describió como ``obvio'' o ``evidente'', eso le deja una mala experiencia de lectura.

Se debe usar ``para todo'', ``para cada'', ``para algún'' o ``para al menos uno'' en lugar de ``para cualquier''.
Frases como ``para todo'', ``para cada'' y ``para cualquier'' se consideran como equivalentes en matemáticas, pero su uso puede resultar en ambigüedades.
P. ej., considere la oración ``se dice que el conjunto \(S\) satisface la propiedad \(p\) si \(p\) se satisface para cualquier elemento \(s\in S\)''.
En este caso, no queda claro si \(p\) debe cumplirse para \emph{todos} los elementos de \(S\) o solamente para \emph{uno} de ellos (sin importar cuál).
Es por esto que es preferible evitar el uso de ``para cualquier'' y usar frases más explicitas.

Es incorrecto utilizar las frases ``como'' y ``debido a que'' en conjunto con ``entonces''.
P. ej., en lugar de escribir:
\[
  \text{Como }n^{2}\text{ es par, entonces }n\text{ también es par.}
\]
es preferible escribir cualquiera de estas:
\[
  \text{Si }n^{2}\text{ es par, entonces }n\text{ también es par.}
\]
\[
  \text{Como }n^{2}\text{ es par, se tiene que }n\text{ también es par.}
\]
\[
  \text{Como }n^{2}\text{ es par, }n\text{ también es par.}
\]
Lo mismo aplica para la frase ``debido a que'', ya que es intercambiable con ``como''.

Se debe evitar usar las frases ``por lo tanto'', ``lo que implica'', ``como consecuencia'', ``por ende'', etc. de forma muy repetitiva.
El propísito de estas frases es establecer alguna conclusión sobre algún razonamiento presentado previamente.
Un error común es utilizar estas frases para encadenar varias proposiciones en un mismo razonamiento.
En estos casos es mejor utilizar frases como ``después'', ``luego'', ``entonces'', etc. o utilizar un estilo de redacción completamente diferente.

%% Esta parte debe moverse a otro capítulo, donde comience a hablarse de demostraciones. 
% \section{Redacción de demostraciones}

% Una demostración es un razonamiento que pretende convencer a otras personas adiestradas en matemáticas que una proposición determinada es verdadera\sidenote{\cite{bloch_2011} p. 48}.
% Una demostración está compuesta por las sig. partes\sidenote{\cite{skiena_2012} pp. 11-12}.
% Primero, al iniciar la demostración, se debe declarar precisa y claramente qué es lo que se pretende demostrar.
% Luego, se deben describir las suposiciones que llevarán a cabo durante la realización de la demostración.
% Si la demostración es corta y sencilla, también es conveniente declarar la estrategia de demostración que se va a utilizar.
% Finalmente, la demostración debe concluir con un pequeño cuadrado negro (\(\blacksquare\)) o las iniciales QED.

%% Texto original:
% Una demostración matemática apropiada consta de tres cosas: una declaración precisa sobre lo que se desea demostrar, una colección de suposiciones (esto es, afirmaciones que se considera que son verdaderas) y una cadena de razonamientos que parten de las suposiciones y terminan en lo que se desea demostrar.


\section{Consejos finales}

Se debe considerar la audiencia para la cuál se está escribiendo para elegir el vocabulario y estilo de redacción más adecuados.
También se debe tener en cuenta cuál es el conocimiento previo que se supone que tiene el lector a la hora de decidir cuáles detalles incluir en el texto y cuáles omitir.

La escritura es un proceso iterativo; rara vez se puede escribir algo correctamente en el primer intento. 
Cada vez que se hace alguna corrección grande o se agrega información faltante al texto, se recomienda volver a leerlo \emph{desde el principio}.
Esto con el propósito de detectar alguna otra corrección que falte hacer y para verificar que los cambios realizados son consistentes con el resto del texto.
También es recomendable dejar de trabajar en el texto por un tiempo (quizás unos días) antes de volverlo a revisar.
Esto permite que cada revisión se haga con una nueva mentalidad y actitud.

Nunca se debe entregar o publicar un texto sin antes permitir que alguien más lo lea y proporcione retroalimentación.
Además, toda corrección realizada al texto debe ser verificada por la misma persona que sugirió dicha corrección. 

\marginnote[-1\baselineskip]{\textbf{Literatura consultada}: \textcite{chartrand_2012} pp. 1-13.}
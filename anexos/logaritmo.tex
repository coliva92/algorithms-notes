\chapter{El logaritmo}

Sea \(x\in\mathbb{R}\) mayor a cero y denominado \emph{argumento} y sea \(b\in\mathbb{R}\) distinto de 1 y denominado \emph{base}.
El \emph{logaritmo} de \(x\) con respecto a \(b\), denotado como \(\log_{b}x\), es la potencia a la que hay que elevar \(b\) para obtener \(x\) como resultado; esto es, \(y=\log_{b}x\) si y sólo si \(b^{y}=x\), donde \(y\in\mathbb{R}\).

\begin{table}[h]
  \caption{Diferentes tipos de logaritmos, su base y su notación correspondiente.}
  \begin{center}
    \begin{tabular}{ccc}
      \toprule
      Tipo de logaritmo & Base & Notación \\
      \midrule
      Decimal & 10 & \(\log{x}\) \\
      Natural & \(e\) & \(\ln{x}\) \\
      Binario & 2 & \(\lg{x}\) \\
      \bottomrule
    \end{tabular}
  \end{center}
\end{table}

\begin{prop}
  La base de un logaritmo no puede ser 0 ni 1.
\end{prop}
  
Debido a que \(0^x=0\) para toda \(x>0\) y debido a que \(1^x=1\) para toda \(x\), los logaritmos con base 0 y 1 no tienen una solución definida.

\begin{prop}
  La base de un logaritmo no puede ser negativa.
\end{prop}

De lo contrario, dependiendo del argumento del logaritmo, la solución podría ser un número complejo.

\begin{prop}
  El argumento de un logaritmo no puede ser negativo.
\end{prop}

Debido a que todo número positivo elevado a cualquier potencia siempre resulta en un número positivo, el logaritmo (con base positiva) de un número negativo no tendría solución.

\section{Leyes de los logaritmos}

Sean \(b,v,x,y\in\mathbb{R}\), donde los valores \(x\) y \(y\) son mayores a cero y los valores \(b\) y \(v\) son distintos de 1. Se tienen las sig. identidades:

\marginnote[1\baselineskip]{La identidad 7 se conoce como la ley de cambio de base. Esta ley implica que \(\log_v{x}=\log_v{b}\cdot\log_b{x}\); esto es, para cambiar la base de un logaritmo, basta con multiplicarlo por un factor constante. Debido a esto, no importa la base de un logaritmo, su comportamiento asintótico siempre será el mismo y es por esta razón que la base del logaritmo suele omitirse al utilizar la notación asintótica.}

\begin{enumerate}
  \begin{multicols}{2}
    \item \(\log_b{1}=0\)
    \item \(\log_b{b}=1\)
    \item \(\log_b(xy)=\log_b{x}+\log_b{y}\)
    \item \(\log_b(x/y)=\log_b{x}-\log_b{y}\)
    \item \(\log_bx^y=y\log_b{x}\)
    \item \(\log_b\sqrt[y]{x}=1/y\cdot\log_b{x}\)
    \item \(\log_b{x}=\log_v{x}/\log_v{b}\)
    \item \(\log_b{x}=1/\log_x{b}\), si \(x\neq 1\)
    \item \(x^{\log_b{y}}=y^{\log_b{x}}\)
  \end{multicols}
\end{enumerate}

Las identidades anteriores se justifican de las manipulaciones algebraicas que se presentan a continuación. Considérese que \(p,q\in\mathbb{R}\) tales que \(x=b^p\) y \(y=b^q\), lo que implica que \(p=\log_b{x}\) \& \(q=\log_b{y}\) (por la definición del logaritmo):
\begin{flalign}
  \text{Es consecuencia directa del hecho que } b^0=1\text{ para cualquier }b.
\end{flalign}
\begin{flalign}
  \text{Es consecuencia directa del hecho que } b^1=b\text{ para cualquier }b.
\end{flalign}
\begin{align}
  xy &= b^pb^q &&\\
  xy &= b^{p+q} \nonumber&&\\
  \log_b(xy) &= p+q \nonumber&&\\
  \log_b(xy) &= \log_b{x}+\log_b{y} \nonumber
\end{align}
\begin{align}
  x/y &= b^p/b^q &&\\
  x/y &= b^{p-q} \nonumber&&\\
  \log_b(x/y) &= p-q \nonumber&&\\
  \log_b(x/y) &= \log_b{x}-\log_b{y} \nonumber
\end{align}
\begin{align}
  x &= b^p &&\\
  x^y &= (b^p)^y \nonumber&&\\
  x^y &= b^{py} \nonumber&&\\
  \log_b{x^y} &= py \nonumber&&\\
  \log_b{x^y} &= y\log_b{x} \nonumber
\end{align}
\begin{align}
  x &= b^p &&\\
  \sqrt[y]{x} &= \sqrt[y]{b^p} \nonumber&&\\
  \sqrt[y]{x} &= b^{p/y} \nonumber&&\\
  \log_b \sqrt[y]{x} &= p/y \nonumber&&\\
  \log_b \sqrt[y]{x} &= 1/y\cdot \log_b{x} \nonumber
\end{align}
\begin{align}
  b^p &= x &\\
  \log_v{b^p} &= \log_v{x} \nonumber&&\\
  p\log_v{b} &= \log_v{x} \nonumber&&\\
  p &= \dfrac{\log_v{x}}{\log_v{b}} \nonumber&&\\
  \log_b{x} &= \log_v{x}/\log_v{b} \nonumber
\end{align}
\begin{align}
  \log_b{x} &= \dfrac{\log_v{x}}{\log_v{b}} &&\\
  \log_b{x} &= \dfrac{\log_x{x}}{\log_x{b}} \nonumber&&\\
  \log_b{x} &= \dfrac{1}{\log_x{b}} \nonumber
\end{align}
\begin{align}
  x^{\log_b{y}}=x^q=(b^p)^q=b^{pq}=(b^q)p=y^p=y^{\log_b{x}}
\end{align}

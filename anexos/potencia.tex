\chapter{La potencia}

Sea \(x\in\mathbb{R}\) distinto de cero y denominado \textbf{base} y sea \(n\in\mathbb{N}\) denominado \textbf{exponente}. 
La \textbf{n-ésima potencia} de \(x\), denotada como \(x^n\), es el producto de \(n\) bases \(x\); esto es, \(x^n\triangleq x\cdot x\cdot\dots(n\text{ veces})\).

\section{Leyes de los exponentes}

Sean \(x,y\in\mathbb{R}\) distintos de cero y sean \(m,n\in\mathbb{N}\), se tienen las sig. identidades:
\begin{enumerate}
  \begin{multicols}{3}
    \item \(x^{0}\triangleq1\)
    \item \(x^{1}\triangleq x\)
    \item \(x^{-1}\triangleq1/x\)
    \item \(x^{m/n}\triangleq\sqrt[n]{x^{m}}\)
    \item \(x^{m}x^{n}=x^{m+n}\)
    \item \(x^{m}/x^{n}=x^{m-n}\)
    \item \((x^{m})^{n}=x^{mn}\)
    \item \((xy)^{n}=x^{n}y^{n}\)
    \item \((x/y)^{n}=x^{n}/y^{n}\)
  \end{multicols}
\end{enumerate}

\newpage
Estas identidades se justifican de las sig. manipulaciones:
\begin{flalign}\tag{5}
  x^mx^n &= x\cdot x\cdot\dots (m\text{ veces})\cdot x\cdot x\dots (n\text{ veces}) = x^{m+n}
\end{flalign}
\begin{flalign}\tag{6}
  \text{\textit{Caso 1}} &: \text{ Si }m>n\text{, entonces} \nonumber&&\\
  x^m/x^n &= \dfrac{x\cdot x\cdot\dots (m\text{ veces})}{x\cdot x\cdot\dots (n\text{ veces})} \nonumber&&\\
  &= 1\cdot 1\cdot\dots(n\text{ veces})\cdot x\cdot x\cdot\dots(m-n\text{ veces}) \nonumber&&\\
  &=x^{m-n} \nonumber&&\\
  \text{\textit{Caso 2}} &: \text{ Si }m=n\text{, entonces} \nonumber&&\\
  x^m/x^n &=1\cdot 1\cdot\dots (m=n\text{ veces}) \nonumber&&\\
  &=x^0 \nonumber&&\\
  &=x^{m-n}\nonumber&&\\
  \text{\textit{Caso 3}} &: \text{ Si }m<n\text{, entonces} \nonumber&&\\
  x^m/x^n &=1\cdot 1\cdot\dots(m\text{ veces})\cdot 1/x\cdot 1/x\cdot\dots(n-m\text{ veces}) \nonumber&&\\
  &=1/x^{n-m} \nonumber&&\\
  &=x^{m-n} \nonumber
\end{flalign}
\begin{flalign}\tag{7}
  (x^m)^n &= [x\cdot x\cdot\dots (m\text{ veces})]\cdot[x\cdot x\cdot\dots (m\text{ veces})]\dots[n\text{ veces}] &&\\
  &= x^{mn}\nonumber
\end{flalign}
\begin{flalign}\tag{8}
  (xy)^n &= xy\cdot xy\cdot\dots (n\text{ veces}) &&\\
  &= x\cdot x\cdot\dots (n\text{ veces})\cdot y\cdot y\cdot\dots (n\text{ veces}) \nonumber&&\\
  &= x^ny^n\nonumber
\end{flalign}
\begin{flalign}\tag{9}
  (x/y)^n &= \dfrac{x}{y}\cdot\dfrac{x}{y}\cdot\dots (n\text{ veces}) = x^n/y^n
\end{flalign}

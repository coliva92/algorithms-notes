\chapter{Redacción Matemática}

A diferencia de otras áreas de la ciencia, el estudio de las matemáticas
no se dedica a la recolección de evidencia física, sino a la manipulación
de conceptos abstractos y la presentación de argumentos lógicos. Es
por esto que es importante aprender a redactar bien en matemáticas y 
en otras ramas derivadas de ellas. 
A continuación se proporcionan algunos consejos sobre cómo redactar 
correctamente cualquier tipo de ``trabajo matemático'', i.e. artículos, 
reportes, ensayos, tareas, etc.

\section{Organización del contenido}

Antes de comenzar a elaborar un trabajo escrito, es recomendable preparar
la sig. información:

\begin{itemize}
    \item Los antecedentes y la motivación del trabajo.
    \item Las definiciones y la notación a ser utilizadas.
    \item Los ejemplos a ser incluidos.
    \item Los resultados a ser presentados con sus demostraciones correspondientes
    (posiblemente en borrador). 
    \item Referencias a otros resultados que se pretenden utilizar.
    \item El orden en que todo lo anterior será presentado.
\end{itemize}

\section{Uso apropiado de símbolos}

\begin{itemize}
    \item Nunca se debe iniciar una oración con un símbolo o una ecuación; es 
    preferible iniciar con un sustantivo seguido del símbolo que se pretende utilizar.
    Por ejemplo, en lugar de escribir:
    
    \[
        \pi\text{ es un número irracional.}
    \]
        
    es preferible escribir:
    
    \[
        \text{La constante }\pi\text{ es un número irracional.}
    \]
    
    \item De ser posible, se deben usar palabras en lugar de comas para separar
    aquellos símbolos que no conformen una lista. Por ejemplo, en lugar
    de escribir:
    
    \[
        \text{Al igual que }\pi\text{, }e\text{ es un número irracional.}
    \]
        
    es preferible escribir:
    
    \[
        \text{Al igual que }\pi\text{, la constante }e\text{ es un número irracional.}
    \]

    \item Se debe evitar el uso de símbolos lógicos (como $\forall$,
    $\exists$, $\Rightarrow$, i.a.) si el tema que se está tratando no
    pertenece al área de la lógica. Estos símbolos se utilizan comúnmente 
    para abreviar frases de uso frecuente en matemáticas. Sin embargo, esta 
    práctica sólo es aceptable en escritos informales y debe evitarse en 
    trabajos profesionales.

    \item Se debe evitar el uso de las abreviaciones ``i.e.'' y ``e.g.''
    si se están utilizando símbolos parecidos. Si no se tiene cuidado de cómo 
    se utilizan, pueden resultar en una redacción confusa. Por ejemplo, 
    en lugar de escribir ``las expresiones
    $\sqrt{-1}$ y $\lim_{n\to\infty}\left(1+1/n\right)^{n}$ no son números
    racionales; i.e. $i$ y $e$ son irracionales'' es preferible escribir:
    ``las expresiones $\sqrt{-1}$ y $\lim_{n\to\infty}\left(1+1/n\right)^{n}$
    no son números racionales; esto es, las constantes $i$ y $e$ son
    irracionales''.

    \item Todo número usado como adjetivo o pronombre se debe escribir con letra.
    Por ejemplo, es preferible escribir:
    
    \[
        \text{Hay exactamente dos grupos de orden 4.}
    \]
    \[
        \text{Cincuenta millones de personas no pueden estar equivocadas.}
    \]
    \[
        \text{Hay un millón de enteros positivos menores a 1,000,001.}
    \]
    
    \item No se deben mezclar símbolos con palabras en una misma oración. Por
    ejemplo, en lugar de escribir:
    
    \[
        \text{Todo entero }>1\text{ es primo o compuesto.}
    \]
    
    es preferible escribir:
    
    \[
        \text{Todo entero que mayor a 1 es primo o compuesto.}
    \]

    Otro ejemplo; aunque la sig. oración puede sonar correcta, está mal
    redactada:
    
    \[
        \text{Como }(x-2)(x-3)=0\text{, se tiene que }x=2\text{ o }3.
    \]
    
    En este caso, es preferible escribir:
    
    \[
        \text{Como }(x-2)(x-3)=0\text{, se tiene que }x=2\text{ o }x=3.
    \]

    \item No se deben introducir símbolos si no se van a utilizar. Por ejemplo,
    en la oración ``toda función biyectiva $f$ tiene una inversa'',
    si el símbolo $f$ no se vuelve a utilizar en el texto, entonces
    es preferible omitirlo. 

    \item No se deben usar símbolos sin primero haberlos introducido. Por ejemplo,
    si se tiene la expresión $n=2k+1$ y esta es la primera vez que aparece
    el símbolo $k$, entonces es preferible redactar esta expresión como:
    
    \[
        \text{Sea }k\text{ un entero, se tiene que }n=2k+1.
    \]
    \[
        \text{Se tiene que }n=2k+1\text{, donde }k\text{ es un entero.}
    \]
    
    \item Se deben seguir convenciones tradicionales sobre el uso de
    algunos símbolos. Estas convenciones se vienen utilizando en la
    desde hace mucho tiempo y mucha gente está familiarizadas con ellas. 
    Debido a ello, uno debe apegarse a ellas. Algunas de estas convenciones son:

    \begin{itemize}
        \item Las letras $a$, $b$ y $c$ representan constantes. 
        \item Las letras $x$, $y$ y $z$ representan variables. 
        \item Las letras $i$ y $j$ representan índices.
        \item Las letras $f$, $g$ y $h$ representan funciones. 
        \item Las letras mayúsculas representan conjuntos o matrices.
    \end{itemize}
    
    \item Se deben utilizar parejas apropiadas de símbolos. Por ejemplo, usar
    $a$ junto con $b$ o utilizar $x$ junto con $y$, etc.

\end{itemize}

\section{Escritura de expresiones algebraicas}

Como regla general, si una expresión algebraica es relativamente corta,
entonces puede escribirse en el mismo renglón que el resto del texto.
De lo contrario, es preferible dedicarles su propio renglón, para evitar 
que una expresión larga tenga que separarse en varios renglones. 

En el caso de manipulaciones algebraicas extensas, es 
recomendable saltar de renglón cada vez que se introduce
un símbolo de comparación. También es recomendable alinear todos los
símbolos de comparación en una misma columna. 

En aquellos casos donde sea inevitable separar una expresión en varios 
renglones, es preferible hacerlo de tal forma que el primer renglón termine 
con un símbolo de operación y el siguiente comience con un término. 
De esta forma, queda indicado más explícitamente que la expresión en el primer 
renglón está incompleta y que continúa en el siguiente. 

\section{Uso apropiado de frases comunes}

\begin{itemize}

    \item Se debe utilizar el artículo ``se'' en lugar de los pronombres ``yo'',
    ``nosotros'' y ``uno''. Usar ``yo'' se considera egocéntrico,
    a menos que se hable sobre alguna experiencia personal. Usar ``nosotros''
    es adecuado, pero se corre el riesgo de sonar demasiado informal. El
    uso de ``uno'' es posible e incluso preferible en algunos casos,
    pero en otros puede ser gramaticalmente incorrecto. El uso del artículo 
    ``se'', seguido de un verbo conjugado adecuadamente, es una forma de redacción 
    muy flexible y por eso se recomienda su uso sobre los pronombres. 
    Por ejemplo, considere las sig. oraciones y compare cuál suena más apropiado:
    
    \begin{itemize}
        \item Ahora demostraré que $n$ es par.
        \item Ahora demostraremos que $n$ es par.
        \item Ahora, uno demostrará que $n$ es par.
        \item Ahora se demostrará que $n$ es par.
    \end{itemize}
    
    \item Se debe evitar el uso de palabras como ``claramente'', ``obviamente'',
    ``evidentemente'', i.a. A menos de que se trate de información casi en extremo
    básica, nunca se debe suponer que lo que se está 
    describiendo queda completamente claro para el lector.
    
    \item Se debe tener cuidado de cómo se utilizan las frases ``para cualquier'',
    ``para algún'' y ``para todo''. Estas frases se consideran como
    equivalentes en la redacción matemática pero, si no se tiene
    cuidado del contexto en que se usan, pueden resultar en ambigüedades.
    Por ejemplo, considere la sig. oración: ``Se dice que el conjunto
    $S$ satisface la propiedad $P$ si $P$ se satisface para cualquier
    elemento $s\in S$''. En este caso, queda ambigüo si la propiedad
    $P$ se debe cumplir para todos los elementos de $S$ o solamente
    para uno. En situaciones como esta, es preferible evitar completamente
    la frase ``para cualquier'' en favor de una redacción más explícita;
    e.g. utilizando frases como ``para todo/cada'', ``para algún''
    o ``para al menos uno''.
    
    \item Es incorrecto utilizar la palabra ``como'' en conjunto con ``entonces''.
    Por ejemplo, en lugar de escribir:
    
    \[
        \text{Como }n^{2}\text{ es par, entonces }n\text{ también es par.}
    \]
    
    es preferible escribir cualquiera de estas:

    \[
        \text{Si }n^{2}\text{ es par, entonces }n\text{ también es par.}
    \]
    \[
        \text{Como }n^{2}\text{ es par, se tiene que }n\text{ también es par.}
    \]
    \[
        \text{Como }n^{2}\text{ es par, }n\text{ también es par.}
    \]
    
    \item Se debe tener cuidado de cómo se utilizan frases como ``por lo tanto'',
    ``lo que implica'', ``como consecuencia'', etc. El propósito de estas frases
    es establecer alguna conclusión sobre algún argumento o razonamiento previamente
    presentado. Un error común es utilizar estas frases para encadenar varios
    razonamientos en un argumento lógico. En este caso, es más apropiado
    usar frases como ``después'', ``luego'', ``entonces'', etc.

\end{itemize}

\section{Consejos generales}

\begin{itemize}
    \item Se deben escribir oraciones completas y que sean gramatical y ortográficamente
    correctas. También se recomienda que la redacción sea lo más simple y concisa posible.
    \item La escritura es un proceso iterativo; rara vez se puede escribir algo
    correctamente en el primer intento. Es por ello que se recomienda
    releer \emph{desde el principio} el trabajo que se está escribiendo
    cada vez que se le hacen varias correcciones. También es recomendable
    dejar descansar el escrito unos días antes de volver a revisarlo. 
    \item A la hora de escribir, se debe considerar la audiencia para la cual
    se está escribiendo; esto es, se debe elegir el vocabulario y estilo
    de redacción más adecuados y se debe considerar el conocimiento
    previo del lector al decidir cuánta información incluir y cuánta omitir. 
    \item Nunca se debe entregar o publicar un trabajo escrito sin antes permitir
    que alguien más lo lea y proporcione retroalimentación. Además, toda
    corrección realizada al escrito debe ser verificada por la misma persona
    que sugirió dicha corrección.
\end{itemize}

\section*{Notas bibliográficas}

Material consultado:
\begin{itemize}
    \item \textcite{chartrand_mathematical_2012}, págs. 1-13.
\end{itemize}

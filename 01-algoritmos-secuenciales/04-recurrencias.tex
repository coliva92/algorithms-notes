
\chapter{Relaciones de Recurrencia}

Una \emph{relación de recurrencia} (o, simplemente, \emph{recurrencia})
es una función expresada en términos de sí misma, pero con una entrada
menor. Las recurrencias se presentan al analizar la eficiencia de
algoritmos de tipo \emph{divide y vencerás}. La \emph{forma cerrada}
de una recurrencia es una función equivalente, pero expresada sin
recursividad. Resolver una recurrencia implica encontrar su forma
cerrada.

En el contexto de los algoritmos divide y vencerás, las recurrencias
suelen tener la sig. forma general: 

\[
T(n)=\begin{cases}
\sum_{i=1}^{p}T(n_{i})+f(n) & \text{para }n>b\\
g(n) & \text{en caso contrario}
\end{cases}
\]
donde 
\begin{itemize}
\item $b\in\mathbb{N}$ es la condición de paro (boundary condition); esto
es, el valor de $n$ requerido para entrar al caso base, 
\item $g:\mathbb{N}\to\mathbb{N}$ es el tiempo de ejecución del caso base, 
\item $p\in\mathbb{N}$ es la cantidad de subproblemas en las que se dividió
la entrada, 
\item $n_{i}\in\mathbb{N}$, tal que $n_{i}<n$, es el tamaño de la entrada
para el subproblema $i$, 
\item $f:\mathbb{N}\to\mathbb{N}$ es el tiempo de ejecución para dividir
la entrada en y/o combinar las soluciones de los $p$ subproblemas 
\end{itemize}
La función $g$ suele omitirse cuando es constante. 

A continuación se presentan tres métodos diferentes para resolver
recurrencias. Para ver ejemplos de estos métodos, se recomienda consultar
el libro de Cormen et al. (2009).

\section{El método maestro}

El método maestro consiste simplemente en aplicar el sig. teorema.
\begin{thm}[Teorema maestro]
Sean $a,n\in\mathbb{N}$ y $b\in\mathbb{R}$, donde $a$ y $b$ son
constantes y $b>1$. Sea $f:\mathbb{N}\to\mathbb{N}$ una función
asintóticamente positiva y sea $c=\log_{b}a$. Si $T:\mathbb{N}\to\mathbb{N}$
es una recurrencia de la forma $T(n)=aT(n/b)+f(n)$, entonces se puede
resolver de alguna de las sig. maneras:
\begin{enumerate}
\item Si $f=O(n^{c-\varepsilon})$ para alguna constante real $\varepsilon>0$,
entonces $T=\Theta(n^{c})$.
\item Si $f=\Theta(n^{c})$, entonces $T=\Theta(n^{c}\log n)$.
\item Si $f=\Omega(n^{c+\varepsilon})$ y si, además, $af(n/b)\leq kf(n)$
para alguna constante real $k<1$ y todo valor de $n$ pasado algún
umbral, entonces $T=\Theta(f)$.
\end{enumerate}
\end{thm}
\begin{rem}
Para el teorema maestro, el término $n/b$ también se puede interpretar
como $\lceil n/b\rceil$ o $\lfloor n/b\rfloor$.
\end{rem}
La principal desventaja del teorema maestro es que no puede aplicarse
a cualquier tipo de recurrencia. La demostración del teorema maestro
se puede encontrar en el libro de Cormen et al.

\section{El método del árbol recursivo}

El método del árbol recursivo es un método gráfico e informal que
consiste de construir un árbol donde cada sub-árbol representa el
tiempo requerido para resolver un subproblema en la recurrencia. Al
sumar el tiempo total de cada nivel del árbol se obtiene la forma
cerrada. 

Antes de estudiar el método en sí, es importante recordar las sig.
propiedades de los árboles $k$-arios.
\begin{prop}
Sea $k\in\mathbb{N}_{0}$ y $n\in\mathbb{N}$. La altura de un árbol
$k$-ario de $n$ nodos se calcula como $\lfloor\log_{k}n\rfloor+1$.
\end{prop}
%
\begin{prop}
Sea $i\in\mathbb{N}_{0}$. Suponiendo que la raíz de un árbol $k$-ario
completo se encuentra en el nivel 0, la cantidad de nodos en el nivel
$i$, se calcula como $k^{i}$.
\end{prop}
El método del árbol recursivo consiste de los sig. pasos:
\begin{enumerate}
\item Desglozar la recurrencia en un árbol de tal forma que la raíz representa
el tiempo requerido para combinar y/o dividir el problema original.
Cada nodo interno representa el tiempo requerido para combinar y/o
dividir un subproblema y cada hoja representa el tiempo requerido
por el caso base. 
\item Calcular la altura del árbol.
\item Calcular la cantidad de hojas.
\item Para cada nivel, sumar el tiempo de ejecución de sus nodos. 
\item Sumar el tiempo de ejecución de todos los niveles para obtener la
forma cerrada.
\end{enumerate}

\section{El método de sustitución}

El método de sustitución es un método formal que consiste en resolver
una recurrencia por inducción matemática. Específicamente, este método
consiste de los sig. pasos:
\begin{enumerate}
\item Proponer una cota asintótica que funja como una solución tentativa
para la recurrencia.
\item Utilizar inducción matemática para demostrar que la cota propuesta
es correcta y para encontrar las constantes de dicha cota.
\end{enumerate}
Este método se puede utilizar para calcular tanto una cota superior
como una inferior. A continuación se presentan algunas consideraciones
y consejos que se deben tener en cuenta al trabajar con este método. 
\begin{itemize}
\item Para proponer una buena solución tentativa:
\begin{itemize}
\item Se puede utilizar el método del árbol recursivo para obtener una solución
tentativa y después utilizar el método de sustitución para demostrar
que dicha solución es correcta o para ajustar la cota.
\item Si la recurrencia tiene una forma similar a alguna cuya solución ya
se conoce, se puede proponer esa solución como una solución tentativa.
\item Se pueden proponer dos cotas holgadas, una inferior y una superior,
y ajustarlas gradualmente hasta que converjan en la solución correcta. 
\end{itemize}
\item Diferencias con la inducción matemática:
\begin{itemize}
\item El caso base se realiza hasta el último.
\item La hipótesis inductiva consiste de suponer que la cota se cumple para
$T(n_{i})$.
\item El paso inductivo consiste en sustituir $T(n_{i})$, en la recurrencia
original, con la forma exacta de la cota de la hipótesis inductiva.
Este paso da origen al nombre del método.
\end{itemize}
\item Se debe demostrar algebraicamente que la cota propuesta se cumple
de forma exacta. Es incorrecto aplicar la notación asintótica en el
paso inductivo para deshacerse de constantes o términos problemáticos.
\item Cuando la cota popuesta es lo más ajustada posible, pero aún así la
inducción no converge, en lugar de proponer una cota más holgada,
se puede proponer una nueva hipótesis inductiva cuya única diferencia
con la anterior es que se le restan los términos de menor grado.
\end{itemize}

\section{Cambio de variable}

En ocasiones, se puede aplicar álgebra para transformar una recurrencia
en alguna otra cuya solución ya se conozca, eliminando por completo
la necesidad de recurrir a cualquiera de los métodos anteriores.

\section*{Notas bibliográficas}

En las págs. 112 y 113 del libro de Cormen et al. se proporcionan
referencias a otros métodos que existen para resolver diferentes tipos
de recurrencias. Uno de ellos es el método de Akra-Bazzi, el cual
es una generalización del teorema maestro que admite variables contínuas
y permite resolver recurrencias donde los subproblemas varían substancialmente
de tamaño. El método de Drmota y Szpankowski (2013) es una extensión
del método de Akra-Bazzi para trabajar con variables discretas. 
\begin{itemize}
\item Cormen T.H., Leiserson C.E., Rivest R.L. \& Stein C., ``Introduction
to Algorithms'', 3ra ed. (2009), MIT Press. Págs. 83-106 y 112-113.
\item Drmota M. \& Szpankowski W., ``A Master Theorem for Discrete Divide
and Conquer Recurrences'', Journal of the ACM (2013), Art. No. 16.
\end{itemize}


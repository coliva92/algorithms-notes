
\chapter{La Notación Asintótica}

La notación asintótica (también conocida como la \emph{notación de
Landau}) es una notación matemática que se utiliza para caracterizar,
por medio de una cota, la tasa de crecimiento de una función cuando
la variable independiente tiende a infinito.

\section{Definiciones básicas}

Sea $n\in\mathbb{N}$ y sean $f:\mathbb{N}\to\mathbb{R}$ y $g:\mathbb{N}\to\mathbb{R}$
dos funciones \emph{asintóticamente positivas}; i.e. $f(n)$ y $g(n)$
siempre son positivos a partir de algún valor de $n$. A continuación
se presentan las definiciones básicas para la notación asintótica.
Obsérvese que, en la notación asintótica, se utiliza el símbolo $=$
en lugar de $\in$ para denotar que una función determinada pertenece
a alguno de los conjuntos introducidos.
\begin{defn}[O grande]
Denotado como $O(g)$, es el conjunto de todas aquellas funciones
$f$ para las cuales existen dos constantes $c\in\mathbb{R}^{+}$
y $n_{0}\in\mathbb{N}$ tales que $0\le f(n)\leq c\cdot g(n)$, para
toda $n\geq n_{0}$. 
\end{defn}
%
\begin{defn}[$\Omega$ grande]
Denotado como $\Omega(g)$, es el conjunto de todas aquellas funciones
$f$ para las cuales existen dos constantes $c$ y $n_{0}$ tales
que $0\le c\cdot g(n)\leq f(n)$, para toda $n\geq n_{0}$.
\end{defn}
%
\begin{defn}[$\Theta$ grande]
Denotado como $\Theta(g)$, es el conjunto de todas aquellas funciones
$f$ para las cuales existen tres constantes $c_{1},c_{2}\in\mathbb{R}^{+}$
y $n_{0}$ tales que $0\le c_{1}\cdot g(n)\leq f(n)\leq c_{2}\cdot g(n)$,
para toda $n\geq n_{0}$.
\end{defn}
\begin{prop}
Se tiene que $f=\Theta(g)$ si y sólo si $f=O(g)$ y $f=\Omega(g)$.
\end{prop}
\begin{figure}[H]
\begin{centering}
\includegraphics[width=1\textwidth]{01-algoritmos-secuenciales/figuras/o-grande}
\par\end{centering}
\caption{{\small{}Ejemplos de cómo interpretar gráficamente las notaciones
$\Theta$-grande, O-grande y $\Omega$-grande. (a) La notación $f=\Theta(g)$
implica que la función $g$ acota $f$ tanto por arriba como por abajo;
esto es, $g$ es una cota ajustada para $f$. (b) La notación $f=O(g)$
implica que $g$ acota $f$ por arriba; esto es, $g$ es una cota
superior para $f$. (c) La notación $f=\Omega(g)$ implica que $g$
acota $f$ por abajo; esto es, $g$ es una cota inferior para $f$.}}
\end{figure}

\begin{defn}[o chica]
Denotado como $o(g)$, es el conjunto de todas aquellas funciones
$f$ tales que, para toda constante $c$, existe una constante $n_{0}$
tal que $0\leq f(n)<c\cdot g(n)$ para toda $n\geq n_{0}$.
\end{defn}
%
\begin{defn}[$\omega$ chica]
Denotado como $\omega(g)$, es el conjunto de todas aquellas funciones
$f$ tales que, para toda constante $c$, existe una constante $n_{0}$
tal que $0\leq c\cdot g(n)<f(n)$ para toda $n\geq n_{0}$.
\end{defn}
\begin{prop}
Si $f=o(g)$, entonces $f=O(g)$.
\end{prop}
%
\begin{prop}
Si $f=\omega(g)$, entonces $f=\Omega(g)$.
\end{prop}

\section{Definiciones a partir de límites}

La notación asintótica también se puede expresar en términos del límite
de la razón de las funciones involucradas (suponiendo que dicho límite
existe). 
\begin{prop}
Se tiene que

\[
\lim_{n\to\infty}\dfrac{f(n)}{g(n)}\in\begin{cases}
[0,\infty) & \text{si y sólo si }f=O(g)\\
(0,\infty] & \text{si y sólo si }f=\Omega(g)\\
(0,\infty) & \text{si y sólo si }f=\Theta(g)
\end{cases}
\]
\end{prop}
%
\begin{prop}
Se tiene que

\[
\text{si }\lim_{n\to\infty}\dfrac{f(n)}{g(n)}=\begin{cases}
0 & \text{entonces }f=o(g)\\
\infty & \text{entonces }f=\omega(g)\\
1 & \text{entonces }f\sim g
\end{cases}
\]
\end{prop}
%
\begin{prop}
Si $f\sim g$, entonces $f=\Theta(g)$.
\end{prop}

\section{Órdenes de crecimiento}

La notación asintótica permite clasificar funciones según su tasa
de crecimiento. Tal clasificación se denomina \emph{orden de crecimiento}
(o, simplemente, \emph{orden}). A continuación se presentan los órdenes
de crecimiento más comunes, listados de aquél que crece más lento
al que crece más rápido.
\begin{enumerate}
\item \emph{Constantes}: $O(1)$
\item \emph{Logarítmicos}: $O(\log n)$
\item \emph{Radicales}: $O(\sqrt{n})$
\item \emph{Lineales}: $O(n)$
\item \emph{Súper lineales}: $O(n\cdot\log n)$
\item \emph{Cuadráticos}: $O(n^{2})$
\item \emph{Cúbicos}: $O(n^{3})$
\item \emph{Exponenciales}: $O(2^{n})$
\item \emph{Factoriales}: $O(n!)$
\end{enumerate}
Se debe tener cuidado sobre cómo interpretar la lista anterior, puesto
que la notación asintótica puede ocultar constantes muy grandes, lo
que puede dar una impresión equivocada de la verdadera tasa de crecimiento
de una función.

\begin{figure}[tb]
\begin{centering}
\includegraphics[width=0.8\textwidth]{01-algoritmos-secuenciales/figuras/ordenes}
\par\end{centering}
\caption{{\small{}Comparación gráfica de los órdenes de crecimiento más frecuentemente
utilizados en las ciencias de la computación.}}
\end{figure}


\section{Propiedades aritméticas}

A continuación se describe cómo hacer aritmética con la notación asintótica.

\subsection{Adición}

La suma de las cotas de dos funciones está gobernada por la función
dominante.

\[
\begin{aligned}O(f)+O(g) & =O(\max\{f,g\})\\
\Omega(f)+\Omega(g) & =\Omega(\max\{f,g\})\\
\Theta(f)+\Theta(g) & =\Theta(\max\{f,g\})\\
o(f)+o(g) & =o(\max\{f,g\})\\
\omega(f)+\omega(g) & =\omega(\max\{f,g\})
\end{aligned}
\]


\subsection{Multiplicación}

La multiplicación de una función con una constante no afecta el comportamiento
asintótico de dicha función.

\[
\begin{aligned}O(c\cdot f) & =O(f)\\
\Omega(c\cdot f) & =\Omega(f)\\
\Theta(c\cdot f) & =\Theta(f)\\
o(c\cdot f) & =o(f)\\
\omega(c\cdot f) & =\omega(f)
\end{aligned}
\]
Cuando dos funciones se multiplican, ambas contribuyen por igual al
comportamiento asintótico de la función resultante.

\[
\begin{aligned}O(f)\cdot O(g) & =O(f\cdot g)\\
\Omega(f)\cdot\Omega(g) & =\Omega(f\cdot g)\\
\Theta(f)\cdot\Theta(g) & =\Theta(f\cdot g)\\
o(f)\cdot o(g) & =o(f\cdot g)\\
\omega(f)\cdot\omega(g) & =\omega(f\cdot g)
\end{aligned}
\]


\section{Comparación de funciones}

La notación asintótica se puede utilizar como un esquema para comparar
dos funciones, similar a como se puede hacer con números reales. Una
forma intuitiva de verlo es trazando las sig. analogías:
\begin{center}
\begin{tabular}{cc}
\toprule 
Números reales & Notación asintótica\tabularnewline
\midrule
$a\leq b$ & $f=O(g)$\tabularnewline
$a\ge b$ & $f=\Omega(g)$\tabularnewline
$a=b$ & $f=\Theta(g)$\tabularnewline
$a<b$ & $f=o(g)$\tabularnewline
$a>b$ & $f=\omega(g)$\tabularnewline
\bottomrule
\end{tabular}
\par\end{center}

\subsection{Transitividad}

Sea $h:\mathbb{N}\to\mathbb{R}$ una función asintóticamente positiva. 
\begin{itemize}
\item Si $f=\Theta(g)$ y $g=\Theta(h)$, entonces $f=\Theta(h)$.
\item Si $f=O(g)$ y $g=O(h)$, entonces $f=O(h)$. 
\item Si $f=\Omega(g)$ y $g=\Omega(h)$, entonces $f=\Omega(h)$.
\item Si $f=o(g)$ y $g=o(h)$, entonces $f=o(h)$.
\item Si $f=\omega(g)$ y $g=\omega(h)$, entonces $f=\omega(h)$.
\end{itemize}

\subsection{Reflexividad}

Las sig. igualdades se satisfacen porque $f(n)=f(n)$ para toda $n$:

\[
\begin{aligned}f & =\Theta(f)\\
f & =O(f)\\
f & =\Omega(f)
\end{aligned}
\]


\subsection{Simetría y simetría traspuesta}

Se tiene que
\begin{itemize}
\item $f=\Theta(g)$ si y sólo si $g=\Theta(f)$.
\item $f=O(g)$ si y sólo si $g=\Omega(f)$.
\item $f=o(g)$ si y sólo si $g=\omega(f)$.
\end{itemize}

\subsection{Carencia de tricotomía}

La tricotomía es la propiedad de los números reales donde, dados dos
números $a$ y $b$, siempre se cumple exactamente una de las sig.
condiciones: $a<b$, $a=b$ o $a>b$. La notación asintótica carece
de esta propiedad, pues\emph{ }se puede dar el caso donde no se cumple
que $f=O(g)$ ni que $f=\Omega(g)$. Por ejemplo, si $f(n)$ oscila
periódicamente en lugar de siempre mantener una tendencia a la alza
o a la baja.

\section*{Notas bibliográficas}
\begin{itemize}
\item Cormen T.H., Leiserson C.E., Rivest R.L. \& Stein C., ``Introduction
to Algorithms'', 3ra ed. (2009), MIT Press. Págs. 43-52.
\item Skiena S.S., ``The Algorithm Design Manual'', 2da ed. (2012), Springer.
Págs. 34-41.
\item \href{https://www.cs.sfu.ca/~ggbaker/zju/math/growth.html}{https://www.cs.sfu.ca/$\sim$ggbaker/zju/math/growth.html}
\end{itemize}


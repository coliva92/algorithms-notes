
\chapter{Análisis de la Exactitud de un Algoritmo}

El primer paso al analizar un algoritmo es determinar si éste es correcto
o incorrecto. El procedimiento que se sigue para llevar a cabo esta
tarea se denomina \emph{análisis de la exactitud} \emph{de un algoritmo}.

\section{Contraejemplo}

Para demostrar que un algoritmo es incorrecto, basta con producir
un \emph{contraejemplo}; i.e. un caso específico para el cual el algoritmo
no termina su ejecución o no produce un resultado que cumpla las características
de salida del problema. Para producir un contraejemplo, se deben explorar
todas las clases de entrada. Se recomienda probar primero con aquellos
casos específicos que representen situaciones extremas o difíciles
de resolver por el algoritmo, o que contengan valores de frontera.
El contraejemplo propuesto debe ser lo más simple y pequeño posible
y debe acompañarse con el resultado incorrecto producido por el algoritmo
y el resultado correcto que debió producir. 

Cabe mencionar que, si no se encontra un contraejemplo para un algoritmo
determinado, eso no implica que dicho algoritmo es correcto.

\section{La invariante de lazo}

Una \emph{invariante de lazo} (loop invariant) es una proposición
lógica que se cumple inmediatamente antes e inmediatamente después
de cada iteración de un bucle determinado. Para producir una invariante
de lazo, se recomienda primero hacer una ``prueba de escritorio''
del bucle. Esto permite identificar patrones en el comportamiento
de dicho bucle que podrían utilizarse para definir una invariante
de lazo. La invariante de lazo siempre se define en función del número
de iteraciones. 

Demostrar que una invariante de lazo se cumple para cualquier iteración
es un procedimiento análogo a una demostración por inducción y consiste
de las sig. etapas: 
\begin{enumerate}
\item \emph{Inicialización}: se considera cuál es el estado del algoritmo
que se tiene antes de comenzar a ejecutar la primera iteración\emph{
}del bucle y se demuestra que la invariante de lazo se cumple bajo
estas condiciones.
\item \emph{Mantenimiento}: se supone que la invariante de lazo se cumple
antes de comenzar alguna iteración genérica $i$ y se determina el
estado del algoritmo que se obtiene como consecuencia de ello. Después,
se ejecuta una iteración del bucle y se demuestra que la invariante
de lazo se cumple antes de comenzar la iteración $i+1$.
\item \emph{Finalización}: se determina cuál es el estado del algoritmo
cuando el bucle termina su ejecución y se utiliza la invariante de
lazo para demostrar que el algoritmo es correcto o para caracterizar
alguna propiedad particular del algoritmo. 
\end{enumerate}

\section{Cómo diseñar algoritmos correctos}

El diseño de un algoritmo es un proceso iterativo; rara vez se puede
diseñar un algoritmo correcto en el primer intento. A continuación
se presentan algunos consejos para diseñar algoritmos correctos. 

\subsection{Definición del problema}

Si el problema no está bien definido, será muy difícil o incluso imposible
diseñar un algoritmo que lo resuelva. Las características de salida
deben ser claras y no deben consistir de objetivos compuestos. En
cuanto a las características de entrada, entre más restrictivas sean,
menor es la dificultad para resolver el problema. Por ende, se recomienda
comenzar por diseñar un algoritmo que admita entradas de características
particulares y después extenderlo a entradas más generales.

\subsection{Analizar la complejidad computacional del problema}

Se recomienda analizar la complejidad computacional del problema al
mismo tiempo que se diseña un algoritmo para resolverlo. De esta forma,
cualquier descubrimiento o avance que se obtenga en un lado podría
aprovecharse después para avanzar en el otro. Además, al final se
obtendría uno de dos posibles resultados: ya sea se llega a la construcción
de un algoritmo correcto o se caracteriza la complejidad del problema.

\subsection{Elegir un modelo adecuado}

Las entidades con las que interactúa un problema en la vida real suelen
representarse por medio de alguna de varias estructuras abstractas
ya conocidas (e.g. cadenas, grafos, conjuntos, etc.). Sin embargo,
las especificaciones del problema no siempre se ajustan perfectamente
a las características de la estructura elegida. En estos casos, se
recomienda ignorar temporalmente aquellos detalles que no encajen
y decidir más adelante, después de trabajar un tiempo con esa estructura,
si dichos detalles son realmente esenciales o no para resolver el
problema. 

\section*{Notas bibliográficas}
\begin{itemize}
\item Cormen T.H., Leiserson C.E., Rivest R.L. \& Stein C., ``Introduction
to Algorithms'', 3ra ed. (2009), MIT Press. Págs. 18-20. 
\item Skiena S.S., ``The Algorithm Design Manual'', 2da ed. (2012), Springer.
Págs. 11-16.
\end{itemize}


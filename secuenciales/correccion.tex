\chapter{Análisis de la corrección de un algoritmo}

El primer paso al analizar un algoritmo es determinar si este es correcto o incorrecto. 
El procedimiento que se sigue para llevar a cabo esta tarea se denomina \emph{análisis de la corrección de un algoritmo}.

\section{El contraejemplo}

\marginnote[0.5\baselineskip]{Si no se encuentra un contraejemplo para un algoritmo determinado, esto no implica que dicho algoritmo es correcto.}

\marginnote[1.5\baselineskip]{Se recomienda buscar un contraejemplo para un algoritmo antes de hacer la demostración de su corrección.}

\marginnote[1.5\baselineskip]{La demostración de la corrección de un algoritmo debe explicar no solo por qué el algoritmo es correcto, sino también por qué no es incorrecto.}

Para demostrar que un algoritmo es incorrecto, basta con producir un \emph{contraejemplo}, esto es, un caso específico para el cual el algoritmo no termina su ejecución o no produce un resultado que cumple las características de salida del problema.
Para encontrar pronto un contraejemplo, se recomienda experimentar con los casos específicos más complicados para el algoritmo; por ej., aquellos que contengan valores empatados o que mezclen valores de frontera de extremos opuestos.
El contraejemplo debe ser lo más simple y pequeño posible, idealmente la razón de por qué el algoritmo es incorrecto debe ser inmediatamente clara. 
Así, una vez encontrado un contraejemplo, se recomienda simplificarlo tanto como sea posible y presentarlo junto con el resultado incorrecto que produjo el algoritmo y el resultado correcto que debió producir.

% Una demostración matemática apropiada consta de tres cosas: una declaración precisa sobre lo que se desea demostrar, una colección de suposiciones (esto es, afirmaciones que se considera que son verdaderas) y una cadena de razonamientos que parten de las suposiciones y terminan en lo que se desea demostrar.

\section{La invariante de lazo}

Una \emph{invariante de lazo} (loop invariant) es una proposición lógica que se cumple inmediatamente antes e inmediatamente después de cada iteración de un bucle o algoritmo iterativo determinado. 
La invariante se define en función de cada iteración \(i\).

Demostrar que una invariantes de lazo se cumple para cualquier iteración es un procedimiento análogo a una demostración por inducción y consta de los sig. pasos: 
\begin{enumerate}
  \item \emph{Inicialización}: se determina cuál es el estado en que se encuentra el algoritmo antes de ejecutar la primera iteración y se demuestra que la invariante de lazo se cumple bajo estas condiciones.
  \item \emph{Mantenimiento}: se supone que la invariante de lazo se cumple antes de comenzar alguna iteración genérica \(i\) y se determina el estado en que se encuentra el algoritmo como consecuencia de ello. 
  Después, se ejecuta la iteración \(i\) y se demuestra que la invariante de lazo se cumple antes de comenzar la iteración \(i+1\).
  \item \emph{Finalización}: se determina cuál es el estado del algoritmo al salir del bucle y se utiliza la invariante de lazo para demostrar que el algoritmo es correcto o para caracterizar alguna propiedad particular de este. 
\end{enumerate}

\section{Recomendaciones para el diseño de algoritmos correctos}

\paragraph*{Definir la salida del problema correctamente.}{
  Las características de salida no deben ser ambiguas y deben describir exactamente cómo determinar si el resultado es correcto.
  Tampoco deben constar de objetivos compuestos (esto es, múltiples objetivos que deben alcanzarse al mismo tiempo).
}

\paragraph*{Típicamente, entre más restrictivas sean las características de entrada, menor es la dificultad para resolver el problema.}{
  Se recomienda comenzar por diseñar un algoritmo correcto que admita entradas con características muy particulares y después extenderlo a entradas más generales. 
}

\paragraph*{Analizar la complejidad computacional del problema al mismo tiempo que se diseña un algoritmo para resolverlo.}{
  De esta forma, cualquier descubrimiento o avance que se obtenga para uno podría aprovecharse después para avanzar en el otro.
  Además, al final se obtendría uno de dos posibles resultados: ya sea se llega a la construcción de un algoritmo correcto o se caracteriza la complejidad computacional del problema.
}

\paragraph*{Modelar el problema de forma adecuada}{
  Las entidades y sus respectivas interacciones en la vida real que constituyen el problema a resolver deben describirse en términos de alguna estructura abstracta ya conocida (cadenas, grafos, conjuntos, entre otros) y sus operaciones correspondientes.
  Sin embargo, las características de estas entidades no siempre se alinean perfectamente con las de la estructura elegida para representarlas.
  En estos casos, se recomienda ignorar temporalmente aquellos detalles que no encajen y decidir más adelante, después de trabajar un tiempo con esa estructura, si dichos detalles son realmente esenciales o no para resolver el problema.
}

\marginnote[-2\baselineskip]{\textbf{Literatura consultada}: \textcite{cormen_introduction_2009}, págs. 18-20; \textcite{skiena_algorithm_2012}, págs. 11-16}

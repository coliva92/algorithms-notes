
\chapter{Conceptos Fundamentales}

\section{Problemas y algoritmos}

Un \emph{problema computacional} (o, simplemente, \emph{problema})
es una relación entre dos conjuntos de entidades, uno de \emph{entrada}
y otro de \emph{salida}. Por ejemplo, el problema de \textsc{Ordenamiento} se
define de la sig. manera:

\begin{itemize}
    \item \emph{Entrada}: una secuencia $A=\{a_1,a_2,\dots,a_n\}$ de $n\in\mathbb{N}$ elementos comparables
    \item \emph{Salida}: una permutación de la secuencia de entrada, $A'=\{a'_1,a'_2,\dots,a'_n\}$, tal que $a'_1\leq a'_2\leq\dots\leq a'_n$.
\end{itemize}

Un \emph{algoritmo} es una secuencia finita
y ordenada de instrucciones bien definidas que describen el procedimiento
a seguir para transformar la entrada de un problema determinado en
la salida correspondiente. Se dice que un algoritmo es

\begin{itemize}
    \item finito, porque tiene una cantidad determinada de instrucciones y su
    ejecución termina en una cantidad determinada de tiempo,
    \item ordenado, porque, dadas dos instrucciones cualesquiera, no hay ambigüedad
    sobre cuál se ejecuta primero, y
    \item bien definido, porque no hay ambigüedad sobre la interpretación de
    las instrucciones.
\end{itemize}

Un \emph{caso específico} (instance) para algún problema particular
es cualquier conjunto de valores que satisfacen las características
de entrada del problema. Por ejemplo, dos casos específicos para el problema
de \textsc{Ordenamiento} son las secuencias $\{4,7,5,1\}$ y $\{d,x,j,e\}$. Los casos específicos pueden agruparse en
diferentes \emph{clases de entrada}, que son todas las diferentes
formas en las que pueden presentarse los casos específicos. Por ejemplo,
supóngase que se tiene un algoritmo que admite como entrada dos valores
numéricos $a,b\in\mathbb{R}$. Este algoritmo tendría tres posibles
clases de entrada: $a<b$, $a=b$ y $a>b$.

Un algoritmo es \emph{correcto} cuando, para cualquier caso específico,
el algoritmo termina su ejecución y produce un resultado que cumple
las características de salida del problema. Se dice que un algoritmo
correcto \emph{resuelve} el problema en cuestión. Un algoritmo es
\emph{incorrecto} cuando existe al menos un caso específico para el
cual el algoritmo no termina su ejecución o termina con un resultado
que no cumple las características de salida. 

Los algoritmos suelen presentarse en forma de un \emph{pseudocódigo}.
Un pseudocódigo es una descripción simple, informal y de alto nivel
de un algoritmo. Comúnmente consiste de estructuras de control de
flujo, notación matemática y/o prosa en lenguaje natural. Se dice
que el pseudocódigo es de alto nivel porque omite detalles de implementación
irrelevantes. A diferencia de un lenguaje de programación, el pseudocódigo
se escribe para que sea leído y entendido por seres humanos, no por
computadoras.

Por último, una \emph{estructura de datos} es un conjunto de procedimientos
para organizar, acceder y manipular una colección de datos.

\section{Análisis de algoritmos}

El \emph{análisis de algoritmos} es el conjunto de técnicas matemáticas
para caracterizar las propiedades particulares de un algoritmo de
forma independiente a su implementación específica en hardware y/o
software. En estos apuntes, el análisis consistirá en describir únicamente
dos características:

\begin{itemize}
    \item La\emph{ exactitud} (correctness); i.e. ¿el algoritmo es correcto?
    \item La\emph{ eficiencia}; i.e. ¿qué tanto crece el tiempo de ejecución
    del algoritmo con respecto al tamaño de la entrada?
\end{itemize}

Se dice que un algoritmo es \emph{eficiente} cuando su tiempo de ejecución
crece polinomialmente con respecto al tamaño de la entrada.

\section*{Notas bibliográficas}

Realmente no existe una definición formal de lo que es un algoritmo;
la ``definición'' presentada aquí es más bien una lista de las características
que posee un algoritmo.

\begin{itemize}
    \item \textcite{cormen_introduction_2009}, págs. 5-14 y 20-22.
    \item \textcite{skiena_algorithm_2011}, págs. 3-13.
    \item \textcite{goodrich_algorithm_2001}, págs. 4, 7 y 8.
\end{itemize}

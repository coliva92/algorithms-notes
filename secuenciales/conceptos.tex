\chapter{Conceptos fundamentales}

Un \emph{problema computacional} es una relación entre dos valores o colecciones de valores: uno de \emph{entrada} (input) y otro de \emph{salida} (output). Por ej., el problema de \textsc{Ordenamiento} se define formalmente de la sig. manera:
\begin{itemize}
  \item \emph{Entrada}: una secuencia de \(n\in\mathbb{N}\) elementos comparables, \(A=\{a_1,a_2,\dots,a_n\}\).
  \item \emph{Salida}: una permutación de la secuencia de entrada, \(A'=\{a'_1,a'_2,\dots,a'_n\}\), tal que \(a'_1\leq a'_2\leq\dots\leq a'_n\).
\end{itemize}
Un \emph{caso específico} (instance) para un problema determinado es cualquier valor o colección de valores que satisfacen la descripción de la entrada del problema. 
Por ej., dos casos específicos para el problema de \textsc{Ordenamiento} son las secuencias \(\{4,7,5,1\}\) y \(\{d,x,j,e\}\).

Informalmente\marginnote{Realmente no existe una definición formal y uniforme sobre qué es un algoritmo.}, un \emph{algoritmo} es un procedimiento inambiguo para transformar la entrada de un problema determinado a la salida correspondiente.
Se dice que un algoritmo es \emph{correcto} cuando, para todo caso específico, el algoritmo termina su ejecución y produce un resultado que cumple la descripción de la salida del problema.
Se dice que un algoritmo correcto \emph{resuelve} el problema en cuestión.

Una \emph{estructura de datos} es una colección de reglas y procedimientos para organizar, accesar y manipular una colección de datos.

Por último, \marginnote{Hay muchas otras características que podrían ser de interés, dependiendo de la aplicación del algoritmo que se está analizando. Por ej., una característica frecuentemente estudiada es el orden de crecimiento de la cantidad de memoria ocupada por el algoritmo con respecto al tamaño de la entrada.} el \emph{análisis de algoritmos} es la colección de técnicas y herramientas matemáticas que se usan para caracterizar las propiedades particulares de un algoritmo determinado de forma independiente a su implementación en hardware y/o software. 
Principalmente, el análisis de un algoritmo consistirá de describir únicamente dos características:

\begin{itemize}
  \item La \emph{corrección}; esto es, ¿el algoritmo es correcto?
  \item La \emph{eficiencia}; esto es, ¿cuál es el orden de crecimiento del 
  tiempo de ejecución del algoritmo con respecto al tamaño de la entrada?
\end{itemize}
Se dice que un algoritmo es \emph{eficiente} si el orden de crecimiento
de su tiempo de ejecución es polinomial.

\marginnote{\textbf{Literatura consultada}: \textcite{cormen_introduction_2009}, págs. 5-14 y 20-22; \textcite{skiena_algorithm_2011}, págs. 3-13.}

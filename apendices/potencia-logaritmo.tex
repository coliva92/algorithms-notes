
\chapter{La Potencia y el Logaritmo}

\section{Potencia}
\begin{defn}[Potencia]
Sean $x\in\mathbb{R}$ y $n\in\mathbb{N}_{0}$, donde $x$ se denomina
\emph{base} y $n$ se denomina \emph{exponente}. La $n$-ésima potencia\emph{
}de $x$, denotada como $x^{n}$, es el producto de $n$ bases $x$.
Esto es, $x^{n}=x\cdot x\cdot\dots\text{ (}n\text{ veces)}$.
\end{defn}
\begin{thm}[Leyes de los exponentes]
\label{laws-of-exp}Sean $x,y\in\mathbb{R}$ y sean $m,n\in\mathbb{N}_{0}$.
Se tienen las sig. identidades sobre potencias:
\begin{enumerate}
\item $x^{0}\triangleq1$
\item $x^{1}\triangleq x$
\item $x^{-1}\triangleq1/x$
\item $x^{m/n}\triangleq\sqrt[n]{x^{m}}$
\item $x^{m}x^{n}=x^{m+n}$
\item $x^{m}/x^{n}=x^{m-n}$, donde $x\neq0$
\item $(x^{m})^{n}=x^{mn}$
\item $(xy)^{n}=x^{n}y^{n}$
\item $(x/y)^{n}=x^{n}/y^{n}$, donde $y\neq0$
\end{enumerate}
\end{thm}
\begin{rem}
Cuando $x=0$, se tiene que $x^{-1}$ es indefinido.
\end{rem}
%
\begin{rem}
Cuando $x=0$, por convención, la identidad $x^{0}=1$ se seguirá
cumpliendo.
\end{rem}
\begin{proof}[Demostración para el teorema \ref{laws-of-exp}]
Las primeras cuatro identidades se aceptan sin requerir una demostración.
El resto se pueden demostrar de forma algebraica. Comenzando por la
identidad 5, ésta se obtiene de:

\[
x^{m}x^{n}=x\cdot x\cdot\dots\text{ (}m\text{ veces)}\cdot x\cdot x\cdot\dots\text{ (}n\text{ veces)}=x^{m+n}
\]
Para la identidad 6 se tienen dos casos:
\begin{casenv}
\item Si $m<n$, se tiene que
\begin{align*}
\dfrac{x^{m}}{x^{n}} & =\dfrac{x\cdot x\cdot\dots\text{ (}m\text{ veces)}}{x\cdot x\cdot\dots\text{ (}n\text{ veces)}}\\
 & =1\cdot1\cdot\dots\text{ (}m\text{ veces)}\cdot\dfrac{1}{x}\cdot\dfrac{1}{x}\cdot\dots\text{ (}n-m\text{ veces)}\\
 & =x^{-1}\cdot x^{-1}\cdot\dots\text{ (}n-m\text{ veces)}\\
 & =x^{m-n}
\end{align*}
\item Si $m\geq n$, se tiene que
\begin{align*}
\dfrac{x^{m}}{x^{n}} & =\dfrac{x\cdot x\cdot\dots\text{ (}m\text{ veces)}}{x\cdot x\cdot\dots\text{ (}n\text{ veces)}}\\
 & =1\cdot1\cdot\dots\text{ (}n\text{ veces)}\cdot x\cdot x\cdot\dots\text{ (}m-n\text{ veces)}\\
 & =x^{m-n}
\end{align*}
\end{casenv}
La identidad 7 se obtiene de:

\[
\left(x^{m}\right)^{n}=\left[x\cdot x\cdot\dots\text{ (}m\text{ veces)}\right]\cdot\left[x\cdot x\cdot\dots\text{ (}m\text{ veces)}\right]\cdot\dots\text{ (}n\text{ veces)}=x^{mn}
\]
La identidad 8 se obtiene de:

\begin{align*}
\left(xy\right)^{n} & =xy\cdot xy\cdot\dots\text{ (}n\text{ veces)}\\
 & =x\cdot x\cdot\dots\text{ (}n\text{ veces)}\cdot y\cdot y\cdot\dots\text{ (}n\text{ veces)}\\
 & =x^{n}y^{n}
\end{align*}
Finalmente, la identidad 9 se obtiene de:

\begin{align*}
\left(\dfrac{x}{y}\right)^{n} & =\dfrac{x}{y}\cdot\dfrac{x}{y}\cdot\dots\text{ (}n\text{ veces)}\\
 & =\dfrac{x\cdot x\cdot\dots\text{ (}n\text{ veces)}}{y\cdot y\cdot\dots\text{ (}n\text{ veces)}}\\
 & =\dfrac{x^{n}}{y^{n}}
\end{align*}
\end{proof}

\section{Logaritmo}
\begin{defn}[Logaritmo]
Sean $b,x\in\mathbb{R}^{+}$, donde $b>1$ y se denomina \emph{base}.
El logaritmo de \emph{$x$} con respecto a \emph{$b$, }denotado como
$\log_{b}x$, es la potencia a la que hay que elevar $b$ para obtener
$x$ como resultado. Esto es, $y=\log_{b}x$ si y sólo si $b^{y}=x$,
donde $y\in\mathbb{R}$.
\end{defn}
%
\begin{defn}[Logaritmo decimal]
Se dice que un logaritmo es decimal cuando su base es 10. El logaritmo
decimal de un número real $x$ se denota como $\log x$. 
\end{defn}
%
\begin{defn}[Logaritmo natural]
Se dice que un logaritmo es natural cuando su base es la constante
de Neper $e$. El logaritmo natural de un número real $x$ se denota
como $\ln x$.
\end{defn}
%
\begin{defn}[Logaritmo binario]
Se dice que un logaritmo es binario cuando su base es 2. El logaritmo
binario de un número real $x$ se denotan como $\lg x$.
\end{defn}
\begin{prop}
\label{no-negative-bases}La base de un logaritmo no puede ser negativa.
\end{prop}
\begin{proof}
Supóngase que $y=\log_{b}x$, donde $b$ es negativo. Esto implica
que $b^{y}=x$ (por la definición del logaritmo). Sin embargo, por
las leyes de los signos, se tiene que el término $b^{y}$ es positivo
cuando $y$ es par y negativo cuando $y$ es impar. Esto implica que
existen valores de $x$ para los cuales $\log_{b}x$ no existe.
\end{proof}
\begin{prop}
\label{no-negative-logarithm}Los números negativos no tienen logaritmo.
\end{prop}
\begin{proof}
Supóngase que $y=\log_{b}x$, donde $x$ es negativo. Esto implica
que $b^{y}=x$ (por la definición del logaritmo). Sin embargo, como
consecuencia de la proposición \ref{no-negative-bases}, se tiene
que $b$ debe ser positiva. Como la potencia de todo número positivo
siempre resulta en un número positivo (por las leyes de los signos),
entonces no existe un número $y$ que satisfaga la relación $b^{y}=x$.
\end{proof}
\begin{prop}
Se tiene que $\log_{b}1=0$ para toda $b>1$.
\end{prop}
\begin{proof}
Supóngase que $y=\log_{b}1$, lo que implica que $b^{y}=1$ (por la
definición del logaritmo). Se tiene entonces lo sig.:

\begin{align*}
b^{y} & =1\\
b^{y} & =b^{0}\\
y & =0\\
\log_{b}1 & =0
\end{align*}
\end{proof}
\begin{prop}
Se tiene que $\log_{b}b=1$ para toda $b>1$.
\end{prop}
\begin{proof}
Supóngase que $y=\log_{b}b$, lo que implica que $b^{y}=b$ (por la
definición del logaritmo). Se tiene entonces lo sig.:

\begin{align*}
b^{y} & =b\\
b^{y} & =b^{1}\\
y & =1\\
\log_{b}b & =1
\end{align*}
\end{proof}
\begin{prop}
El logaritmo de todo número mayor a 1 es positivo. Esto es, $\log_{b}x>0$
para toda $x>1$.
\end{prop}
\begin{proof}
Supóngase que $y=\log_{b}x$, donde $x>1$. Esto implica que $b^{y}=x$
(por la definición del logaritmo). De la proposición (\ref{no-negative-bases})
se tiene que $b$ es positivo. Por lo tanto, para que se satisfaga
la relación $b^{y}=x$, el valor de $y$ debe ser positivo. 
\end{proof}
%
\begin{prop}
El logaritmo de todo número menor a 1 es negativo. Esto es, $\log_{b}x<0$
para toda $x\in(0,1)$.
\end{prop}
\begin{proof}
Supóngase que $y=\log_{b}x$, donde $x\in(0,1)$. Esto implica que
$b^{y}=x$ (por la definición del logaritmo). De las proposiciones
(\ref{no-negative-bases}) y (\ref{no-negative-logarithm}) se tiene
que $b$ y $x$ son ambos positivos. Por lo tanto, para que se satisfaga
la relación $b^{y}=x$, el valor de $y$ debe ser negativo (por las
leyes de los exponentes).
\end{proof}
\begin{thm}[Leyes de los logaritmos]
Sean $b,v,x,y\in\mathbb{R^{+}}$, donde $b,v>1$. Se tienen las sig.
identidades sobre logaritmos:
\begin{enumerate}
\item $\log_{b}\left(x\cdot y\right)=\log_{b}x+\log_{b}y$
\item $\log_{b}\left(x/y\right)=\log_{b}x-\log_{b}y$
\item $\log_{b}x^{y}=y\cdot\log_{b}x$
\item $\log_{b}\sqrt[y]{x}=1/y\cdot\log_{b}x$
\item $\log_{b}x=\log_{v}x/\log_{v}b$
\item $\log_{b}x=1/\log_{x}b$
\item $x^{\log_{b}y}=y^{\log_{b}x}$
\end{enumerate}
\end{thm}
\begin{proof}
Todas las identidades se pueden demostrar algebraicamente. Comenzando
por la identidad 1, sean $\alpha,\beta\in\mathbb{R}$ tales que $\alpha=b^{x}$
y $\beta=b^{y}$. Entonces, se tiene que:

\begin{align*}
\alpha\cdot\beta & =b^{x}b^{y}\\
\alpha\cdot\beta & =b^{x+y}\\
\log_{b}(\alpha\cdot\beta) & =x+y\\
\log_{b}(\alpha\cdot\beta) & =\log_{b}\alpha+\log_{b}\beta
\end{align*}
La identidad 2 se origina de una manipulación algebraica similar:

\begin{align*}
\dfrac{\alpha}{\beta} & =\dfrac{b^{x}}{b^{y}}\\
\dfrac{\alpha}{\beta} & =b^{x-y}\\
\log_{b}\left(\dfrac{\alpha}{\beta}\right) & =x-y\\
\log_{b}\left(\dfrac{\alpha}{\beta}\right) & =\log_{b}\alpha-\log_{b}\beta
\end{align*}
Para la identidad 3, sea $\theta\in\mathbb{R}$ tal que $\theta=\log_{b}x$.
Esto implica que $b^{\theta}=x$ (por la definición del logaritmo),
por lo que se tiene lo sig.:

\begin{align*}
b^{\theta} & =x\\
(b^{\theta})^{y} & =x^{y}\\
b^{\theta y} & =x^{y}\\
\theta\cdot y & =\log_{b}x^{y}\\
y\cdot\log_{b}x & =\log_{b}x^{y}
\end{align*}
La identidad 4 se obtiene de:

\begin{align*}
b^{\theta} & =x\\
\sqrt[y]{b^{\theta}} & =\sqrt[y]{x}\\
b^{\theta/y} & =\sqrt[y]{x}\\
\dfrac{\theta}{y} & =\log_{b}\sqrt[y]{x}\\
\dfrac{1}{y}\cdot\log_{b}x & =\log_{b}\sqrt[y]{x}
\end{align*}
La identidad 5 se obtiene de:

\begin{align*}
b^{\theta} & =x\\
\log_{v}b^{\theta} & =\log_{v}x\\
\theta\cdot\log_{v}b & =\log_{v}x\\
\theta & =\dfrac{\log_{v}x}{\log_{v}b}\\
\log_{b}x & =\dfrac{\log_{v}x}{\log_{v}b}
\end{align*}
La identidad 6 se obtiene de:

\begin{align*}
b^{\theta} & =x\\
\log_{x}b^{\theta} & =1\\
\theta\cdot\log_{x}b & =1\\
\theta & =\dfrac{1}{\log_{x}b}\\
\log_{b}x & =\dfrac{1}{\log_{x}b}
\end{align*}
Finalmente, para la identidad 7, supóngase que $\alpha=\log_{b}x$
y $\beta=\log_{b}y$. Esto implica que $b^{\alpha}=x$ y que $b^{\beta}=y$
(por la definición del logaritmo). Entonces, se tiene que:

\begin{align*}
x^{\log_{b}y} & =x^{\beta}\\
 & =(b^{\alpha})^{\beta}\\
 & =b^{\alpha\beta}\\
 & =(b^{\beta})^{\alpha}\\
 & =y^{\alpha}\\
 & =y^{\log_{b}x}
\end{align*}
\end{proof}
\begin{rem}
Debido a que cambiar la base de cualquier logaritmo altera dicho valor
en un factor constante, el comportamiento asintótico sigue siendo
el mismo sin importar la base. Como consecuencia, la base del logaritmo
suele omitirse al usar notación asintótica.
\end{rem}


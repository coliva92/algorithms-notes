\chapter{La Potencia y el Logaritmo}

\section{Potencia}

Sean $x\in\mathbb{R}$ y $n\in\mathbb{N}_{0}$, donde $x$ se denomina
\emph{base} y $n$ se denomina \emph{exponente}. La $n$-ésima potencia\emph{
}de $x$, denotada como $x^{n}$, es el producto de $n$ bases $x$.
Esto es, $x^{n}=x\cdot x\cdot\dots\text{ (}n\text{ veces)}$.

\begin{thm}[Leyes de los Exponentes]
    \label{laws-of-exp}
    Sean $x,y\in\mathbb{R}$ y sean $m,n\in\mathbb{N}_{0}$.
    Se tienen las sig. identidades sobre potencias:
    \begin{enumerate}
    \begin{multicols}{2}
        \item $x^{0}\triangleq1$
        \item $x^{1}\triangleq x$
        \item $x^{-1}\triangleq1/x$
        \item $x^{m/n}\triangleq\sqrt[n]{x^{m}}$
        \item $x^{m}x^{n}=x^{m+n}$
        \item $x^{m}/x^{n}=x^{m-n}$, donde $x\neq0$
        \item $(x^{m})^{n}=x^{mn}$
        \item $(xy)^{n}=x^{n}y^{n}$
        \item $(x/y)^{n}=x^{n}/y^{n}$, donde $y\neq0$
    \end{multicols}
    \end{enumerate}
\end{thm}

\begin{rem}
    Cuando $x=0$, se tiene que $x^{-1}=0^{-1}$ es indefinido.
\end{rem}

\begin{rem}
    Cuando $x=0$, por convención, la identidad $x^{0}=0^0=1$ se sigue
    cumpliendo.
\end{rem}

\begin{proof}[Demostración para el teorema \ref{laws-of-exp}]
    Las primeras cuatro identidades se aceptan sin requerir una demostración.
    El resto se puede demostrar de forma algebraica. Comenzando por la
    identidad 5, ésta se obtiene de
    
    \[
    x^{m}x^{n}=x\cdot x\cdot\dots\text{ (}m\text{ veces)}\cdot x\cdot x\cdot\dots\text{ (}n\text{ veces)}=x^{m+n}
    \]
    
    Para la identidad 6 se tienen dos casos.
    \begin{casenv}
        \item Si $m<n$, se tiene que
        \begin{align*}
            \dfrac{x^{m}}{x^{n}} &= \dfrac{x\cdot x\cdot\dots\text{ (}m\text{ veces)}}{x\cdot x\cdot\dots\text{ (}n\text{ veces)}}\\
            &= 1\cdot1\cdot\dots\text{ (}m\text{ veces)}\cdot\dfrac{1}{x}\cdot\dfrac{1}{x}\cdot\dots\text{ (}n-m\text{ veces)}\\
            &= x^{-1}\cdot x^{-1}\cdot\dots\text{ (}n-m\text{ veces)}\\
            &= x^{m-n}
        \end{align*}
        \item Si $m\geq n$, se tiene que
        \begin{align*}
            \dfrac{x^{m}}{x^{n}} &= \dfrac{x\cdot x\cdot\dots\text{ (}m\text{ veces)}}{x\cdot x\cdot\dots\text{ (}n\text{ veces)}}\\
            &= 1\cdot1\cdot\dots\text{ (}n\text{ veces)}\cdot x\cdot x\cdot\dots\text{ (}m-n\text{ veces)}\\
            &= x^{m-n}
        \end{align*}
    \end{casenv}
    
    La identidad 7 se obtiene de
    
    \[
    \left(x^{m}\right)^{n}=\left[x\cdot x\cdot\dots\text{ (}m\text{ veces)}\right]\cdot\left[x\cdot x\cdot\dots\text{ (}m\text{ veces)}\right]\cdot\dots\text{ (}n\text{ veces)}=x^{mn}
    \]
    
    La identidad 8 se obtiene de
    
    \begin{align*}
        \left(xy\right)^{n} &=xy\cdot xy\cdot\dots\text{ (}n\text{ veces)}\\
        &= x\cdot x\cdot\dots\text{ (}n\text{ veces)}\cdot y\cdot y\cdot\dots\text{ (}n\text{ veces)}\\
        &= x^{n}y^{n}
    \end{align*}
    
    Finalmente, la identidad 9 se obtiene de
    
    \begin{align*}
        \left(\dfrac{x}{y}\right)^{n} &= \dfrac{x}{y}\cdot\dfrac{x}{y}\cdot\dots\text{ (}n\text{ veces)}\\
        &= \dfrac{x\cdot x\cdot\dots\text{ (}n\text{ veces)}}{y\cdot y\cdot\dots\text{ (}n\text{ veces)}}\\
        &= \dfrac{x^{n}}{y^{n}}
    \end{align*}
\end{proof}

\section{Logaritmo}

Sean $b,x\in\mathbb{R}^{+}$, donde $b\neq 1$ y se denomina \emph{base}.
El logaritmo de $x$ con respecto a $b$, denotado como
$\log_{b}x$, es la potencia a la que hay que elevar $b$ para obtener
$x$ como resultado. Esto es, $y=\log_{b}x$ si y sólo si $b^{y}=x$,
donde $y\in\mathbb{R}$.

Cuando la base es 10, el logaritmo se denomina \emph{logaritmo decimal} y se denota 
como $\log x$. Cuando la base es la constante de Neper $e$, el logaritmo se
denomina \emph{logaritmo natural} y se denota como $\ln x$. Por último, cuando la 
base es 2, el logaritmo se denomina \emph{logaritmo binario} y se denota $\lg x$.

\begin{prop}
    La base de un logaritmo no puede ser 0 ni 1.
\end{prop}

Dado que $0^y=0$ y $1^y=1$ para cualquier valor de $y$,
los logaritmos con base 0 y 1 no tienen una solución definida.

\begin{prop}
    \label{no-negative-bases}
    La base de un logaritmo no puede ser negativa.
\end{prop}

Dado que la raíz cuadrada de un número negativo resulta en un número complejo, 
los logaritmos con base negativa no siempre tienen una solución en el plano real.

\begin{prop}
    \label{no-negative-logarithm}
    El logaritmo no puede aplicarse sobre números negativos.
\end{prop}

Como todo número positivo elevado a cualquier potencia resulta 
en un número positivo, los logaritmos (con base positiva) 
de números negativos no tienen solución.

\begin{prop}
    Se tiene que $\log_{b}1=0$.
\end{prop}

Lo anterior es consecuencia directa del hecho de que $b^0=1$ 
para toda $b\neq 0$.

\begin{prop}
    Se tiene que $\log_{b}b=1$.
\end{prop}

Lo anterior se deriva directamente del hecho de que
$b^1=b$ para cualquier $b$.

\begin{thm}[Leyes de los Logaritmos]
    Sean $b,v,x,y\in\mathbb{R^{+}}$, donde $b,v\neq 1$. Se tienen las sig.
    identidades sobre logaritmos:
    \begin{enumerate}
    \begin{multicols}{2}
        \item $\log_{b}\left(x y\right)=\log_{b}x+\log_{b}y$
        \item $\log_{b}\left(x/y\right)=\log_{b}x-\log_{b}y$
        \item $\log_{b}x^{y}=y\log_{b}x$
        \item $\log_{b}\sqrt[y]{x}=1/y\cdot\log_{b}x$
        \item $\log_{b}x=\log_{v}x/\log_{v}b$ (Cambio de base)
        \item $\log_{b}x=1/\log_{x}b$ (Cambio de base)
        \item $x^{\log_{b}y}=y^{\log_{b}x}$
    \end{multicols}
    \end{enumerate}
\end{thm}

\begin{proof}
    Todas las identidades se pueden demostrar algebraicamente. Comenzando
    por la identidad 1, sean $\alpha,\beta\in\mathbb{R}$ tales que $\alpha=b^{x}$
    y $\beta=b^{y}$. Entonces, se tiene que

    \begin{align*}
        \alpha\beta &= b^{x}b^{y}\\
        \alpha\beta &= b^{x+y}\\
        \log_{b}(\alpha\beta) &= x+y\\
        \log_{b}(\alpha\beta) &= \log_{b}\alpha+\log_{b}\beta
    \end{align*}
    
    La identidad 2 se origina de una manipulación algebraica similar:
    
    \begin{align*}
        \dfrac{\alpha}{\beta} &= \dfrac{b^{x}}{b^{y}}\\
        \dfrac{\alpha}{\beta} &= b^{x-y}\\
        \log_{b}\left(\dfrac{\alpha}{\beta}\right) &= x-y\\
        \log_{b}\left(\dfrac{\alpha}{\beta}\right) &= \log_{b}\alpha-\log_{b}\beta
    \end{align*}
    
    Para la identidad 3, sea $\theta\in\mathbb{R}$ tal que $\theta=\log_{b}x$.
    Esto implica que $b^{\theta}=x$ (por la definición del logaritmo),
    por lo que se tiene lo sig.:

    \begin{align*}
        b^{\theta} &= x\\
        (b^{\theta})^{y} &= x^{y}\\
        b^{\theta y} &= x^{y}\\
        \theta y &= \log_{b}x^{y}\\
        y\log_{b}x &= \log_{b}x^{y}
    \end{align*}
    
    La identidad 4 se obtiene de
    
    \begin{align*}
        b^{\theta} &= x\\
        \sqrt[y]{b^{\theta}} &= \sqrt[y]{x}\\
        b^{\theta/y} &= \sqrt[y]{x}\\
        \dfrac{\theta}{y} &= \log_{b}\sqrt[y]{x}\\
        \dfrac{1}{y}\log_{b}x &= \log_{b}\sqrt[y]{x}
    \end{align*}
    
    La identidad 5 se obtiene de
    
    \begin{align*}
        b^{\theta} &= x\\
        \log_{v}b^{\theta} &= \log_{v}x\\
        \theta\log_{v}b &= \log_{v}x\\
        \theta &= \dfrac{\log_{v}x}{\log_{v}b}\\
        \log_{b}x &= \dfrac{\log_{v}x}{\log_{v}b}
    \end{align*}
    
    La identidad 6 se obtiene a partir de la manipulación anterior, suponiendo
    que $v=x\neq 1$.
    
    \begin{align*}
        \log_{b}x &= \dfrac{\log_{v}x}{\log_{v}b}\\
        \log_{b}x &= \dfrac{\log_{x}x}{\log_{x}b}\\
        \log_{b}x &= \dfrac{1}{\log_{x}b}
    \end{align*}
    
    Finalmente, para la identidad 7, supóngase que $\alpha=\log_{b}x$
    y $\beta=\log_{b}y$. Esto implica que $b^{\alpha}=x$ y que $b^{\beta}=y$
    (por la definición del logaritmo). Entonces, se tiene que
    
    \begin{align*}
        x^{\log_{b}y} & =x^{\beta} = (b^{\alpha})^{\beta}= b^{\alpha\beta}
        = (b^{\beta})^{\alpha}= y^{\alpha}= y^{\log_{b}x}
    \end{align*}
\end{proof}

\begin{rem}
    Por la regla de cambio de base, se tiene que $\log_{v}b\cdot\log_{b}x=\log_{v}x$.
    En otras palabras, ya que $b$ y $v$ son constantes conocidas, cambiar la base de cualquier 
    logaritmo altera dicho valor en un factor constante. Esto implica que el comportamiento
    asintótico de cualquier logaritmo es el mismo sin importar la base. Como 
    consecuencia, la base de un logaritmo suele omitirse al utilizar la notación 
    asintótica.
\end{rem}
